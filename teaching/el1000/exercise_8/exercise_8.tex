\documentclass[12pt]{article}
\usepackage[swedish]{babel}
\usepackage[T1]{fontenc}{
\usepackage[utf8]{inputenc}
\usepackage{amssymb}
\usepackage{amsmath}
\usepackage{amsfonts}
\usepackage[colorinlistoftodos,prependcaption,textsize=small]{todonotes}
\usepackage{url,graphicx,tabularx,array,geometry,color,float,placeins}
\setlength{\parskip}{1ex} %--skip lines between paragraphs
\setlength{\parindent}{0pt} %--don't indent paragraphs
%\setcounter{secnumdepth}0}
\usepackage{tikz}
\usetikzlibrary{shapes, arrows, 3d, decorations, calc, decorations.markings}

\tikzstyle{block} = [draw, rectangle, thick, minimum height=3em, minimum width=2em, align=center, text width=1.5cm, top color=white, bottom color=white!85!black]
\tikzstyle{input} = [coordinate]
\tikzstyle{output} = [coordinate]
\tikzstyle{tmp} = [coordinate]
\tikzstyle{sum} = [draw, circle, node distance=1cm]
\tikzstyle{wire} = [->,thick]
\tikzstyle{lab} = []
\tikzstyle{laba} = [lab, label distance=0cm]
\tikzstyle{labb} = [lab, label distance=0.5em]
\tikzstyle{split} = [fill,black,circle,inner sep=0.03cm]
\tikzset{cross/.style={cross out, draw=black, minimum size=2*(#1-\pgflinewidth), inner sep=0pt, outer sep=0pt},
%default radius will be 1pt. 
cross/.default={5pt}}

\newcommand{\sspan}[1]{\mathrm{span}\left\{#1\right\}}
\newcommand{\ima}[1]{\mathrm{Im}\; #1}
\newcommand{\qline}{\hrule \vspace*{10pt}}

%\linespread{2} %-- Uncomment for Double Space
\begin{document}
\begin{titlepage}
\author{Martin Biel \\ \texttt{mbiel@kth.se}}
\title{EL1000 - Reglerteknik, allmän kurs \\ \Large Övning 8}
\date{19 september 2016}
\end{titlepage}

\maketitle

\section*{Repetition}

\begin{itemize}
\item $\omega_c$ - Skärfrekvens \\
frekvens när förstärkningen är $1$ (Nyquist skär enhetscirkeln) 
\item $\omega_p$ - Fasskärfrekvens \\
frekvens där fasen är $-180^{\circ}$ (Nyquist skär negativa Re-axeln)
\item $A_m$ - Amplitudmarginal \\
skillnad i amplitud mellan förstärkningen vid $\omega_p$ och $1$
\item $\varphi_m$ - Fasmarginal \\
skillnad i fas mellan fasen vid $\omega_c$ och $-180^{circ}$.
\end{itemize}

$\omega_c \sim \frac{1}{T_r} \Rightarrow \omega_c$ relaterar till systemets hastighet
\begin{itemize}
\item $\omega_c$ stor $\rightarrow$ snabbt system
\item $\omega_c$ liten $\rightarrow$ långsamt system
\end{itemize}

$\varphi$ relaterar till systemets stabilitet/dämpning
\begin{itemize}
\item $\varphi > 0 \rightarrow$ stabilt system
\item $\varphi$ stor $\rightarrow$ dämpat system
\item $\varphi$ liten $\rightarrow$ svängigt system 
\end{itemize}

\section*{Lead-lag kompensering}
Regulatorkonstruktion baserat på specifikationer i frekvensplanet:
\begin{itemize}
\item $\varphi$ - stabilitet/dämpning
\item $\omega_c$ - hastighet
\item $e_i$ - felstorlek
\end{itemize}
Lead-lag kompensering är en systematisk metod för att erhålla ett öppet system med önskade värden på $\omega_c$,$\varphi_m$ och $e_i$.

\subsection*{Lead-del}

\[F_{lead}(s) = K \frac{\tau_ds+1}{\beta\tau_ds+1}\]
$K$ ändrar enbart amplitud. Ifall fasmarginalen är tillräcklig vid önskad $\omega_c$ räcker det med en P-regulator. \\

Ifall $\varphi_m$ inte tillräcklig vid $\omega_c$ höjs fasen med en lead-länk
\begin{enumerate}
\item $\beta$ avgör den maximala fashöjningen 
  \[\phi = \arctan{\frac{1-\beta}{2\sqrt{\beta}}} \text{ (Använd diagram på s.106)}\]
Notera att små $\beta$ leder till stor förstärkning av mätbrus.
\item $\tau_D$ bestämmer vid vilken frekvens den maximala fasförskjutningen $\phi$ inträffar 
  \[\tau_D = \frac{1}{\omega_c\sqrt{\beta}}\] 
  $\Rightarrow \phi$ vid $\omega_c$
\item $K$ kan nu väljas så att den önskade $\omega_c$ uppnås 
  \[|F_{lead}(i\omega_c)| = \frac{K}{\sqrt{\beta}}\] 
  \[\Rightarrow K = \frac{\sqrt{\beta}}{|G(i\omega_c)|}\]
  Notera att $|F_{lead}(s)|: K \rightarrow \dfrac{K}{\beta}$ \\
  Ifall den nödvändiga fasadvanceringen är större än ungefär $40^{\circ}$ används två lead-länkar eftersom $\beta$ annars blir för liten. Notera att $K$ förekommer i varje lead-länk och ger upphov till förstärkningen $K^2$.
\end{enumerate}

\subsection*{Lag-del}

\[F_{lag}(s) = \frac{\tau_Is+1}{\tau_Is+\gamma}  \]
Lag-länken införs för att minska stationära fel (integrerande del). $\gamma$ läggs till i nämnaren för att göra lag-länken stabil. En följd av detta är att det stationära felet aldrig fullt elimineras, eftersom det inte längre sker en ren integration. Det gäller dock att 
\[F_{lag}(0) = \frac{1}{\gamma}\]
så att den statiska förstärkningen (och därmed det statiska felet) kan väljas godtyckligt genom att variera $\gamma$.
\begin{enumerate}
\item Välj $\gamma$ så att önskad feldämpning uppnås. Använd slutvärdesstasen för lämplig insignal (steg, ramp, ...) för att undersöka det statiska felet. Kom ihåg att $F_{lead}(s)$ ger upphov till en förstärkning $K$ i $s = 0$. ($|F_{lead}(0)| = K$).
\item $\tau_I$ avgör hur höga frekvenser som förstärks av lag-länken. Effekten av förstärkningen blir en fasretardation. Fasretarderingen vid $\omega_c$ ges av
  \[\mathrm{arg}(F_{lag}(i\omega_c)) = -\arctan{\frac{1}{\tau_I \omega_c}}\]
Sätt tillexempel $\tau_I$ till 
\[\tau_I = \frac{10}{\omega_c}\]
vilket ger upphov till $5.7^{\circ}$ i fasretarderingen. Detta kan kompenseras för genom att höja fasen $\approx 6^{\circ}$ extra. Amplitudförändringen från lag-delen kan försummas, eftersom den inte märkbart kommer flytta $\omega_C$.
\end{enumerate}

Den resulterande regulatorn ges av 
\[F_{lead-lag}(s) = K \frac{\tau_Ds + 1}{\beta\tau_D + 1} \frac{\tau_Is +1}{\tau_Is+\gamma}\]
Lead-lag kompensering är en approximativ metod, så dubbelkolla alltid att alla specifikationer uppfylls!

\section*{Uppgift 5.20}
Bodediagrammet för ett system $G(s)$ är återgivet i Figur \ref{fig:5_20bode}.
\begin{figure}[h!]
  \centering
  \includegraphics[width=0.5\textwidth]{5_20bode}
  \caption{Bodediagram för $G(s)$}
  \label{fig:5_20bode}
\end{figure}
\FloatBarrier

\subsection*{a)}
$G(s)$ återkopplas med en regulator $F(s)$. Bestäm det slutna systemet.
\qline

\[G_c(s) = \frac{F(s)G(s)}{1+F(s)G(s)}\]

\subsection*{b)}
Låt $F(s) = K$. Beräkna $\omega_c$ och $\varphi_m$ för $G_0(s) = KG(s)$ då $K = 1$. Är slutna systemet stabilt för $K = 1$.\vspace*{5pt}
\qline
Från bodediagrammet i Figur \ref{fig:5_20bode} erhålls
\begin{align*}
  \omega_c &= 1 [\mathrm{rad/s}] \\
  \varphi_m &= \arg{G(i\omega_c)}-(-180^{\circ}) = -130^{\circ} - (-180^{\circ}) = 50^{\circ}
\end{align*}
Eftersom $\varphi_m > 0$ är det slutna systemet stabilt då $K = 1$.

\subsection*{c)}
Beräkna $K$ så systemet blir dubbelt så snabbt.
\qline
Minns att $\omega_c \sim \dfrac{1}{T_r}$. \\

$\Rightarrow \omega_{c,d} = 2\omega_c = 2 [\mathrm{rad/s}]$. \\

$|KG(i2)| = 1 \Rightarrow K = \dfrac{1}{|G(i2)|} = \dfrac{1}{0.3} = 3.33$

\subsection*{d)}
Hur försämrades systemet i c)?
\qline
Vid den nya skärfrekvensen är fasen $\approx -175^{\circ}$. Alltså är $\varphi_m = 5^{\circ}$ vilket ger ett mer svängigt system än i a).

\subsection*{e)}
Använd en lead-länk så systemet approximativt bibehåller samma översläng men är dubbelt så snabbt.
\vspace*{5pt}
\qline
Specifikation:
\begin{align*}
  \omega_{c,d} &= 2 [\mathrm{rad/s}] \\
  \varphi_m &= 50^{\circ}
\end{align*}
Eftersom fasen är $5^{\circ}$ vid den önskade skärfrekvensen behöver vi höja fasen med $45^{\circ}$, så att $\phi = 45^{\circ}$. \\
$\Rightarrow \beta = 0.18$ (från diagram). \\ 
\[\tau_D = \frac{1}{\omega_{c,d}\sqrt{\beta}} = \frac{1}{2\sqrt{0.18}} \approx 1.18\] 
\[|F_{lead}(i\omega_{c,d})| \cdot |G(i\omega_{c,d})| = 1 \Rightarrow \frac{K}{\sqrt{0.18}}\cdot 0.3 = 1 \Rightarrow K \approx 1.4 \]

\[\Rightarrow F_{lead}(s) = 1.4 \frac{1.18s+1}{0.21s+1}\]

\subsection*{f)}
Beräkna det statiska felet efter ett enhetssteg.
\qline

\[\lim_{t \to \infty}e(t) = \lim_{s \to 0} sE(s) = \lim_{s \to 0}s \cdot \frac{1}{1+F(s)G(s)} \cdot \frac{1}{s} = \frac{1}{1+KG(0)} = 0.17\]
(För att statiska felet ska gå mot noll krävs att antingen $F(s)$ eller $G(s)$ innehåller en ren integration.)

\subsection*{g)h)}
Inför en lag-länk. Välj $\gamma$ så att det statiska felet blir $0.01$. Välj även ett lämpligt värde på $\tau_I$.
\qline
\[\lim_{t \to \infty}e(t) = \lim_{s \to 0} sE(s) = \lim_{s \to 0}s \cdot \frac{1}{1+F(s)G(s)} \cdot \frac{1}{s} = \frac{1}{1+\frac{K}{\gamma}G(0)} = 0.01\] 
\[\Rightarrow 1+\frac{K}{\gamma}G(0) = 100 \Rightarrow \gamma = 0.048\]
Vi betraktar olika alternativ på $\tau_I$:
\begin{itemize}
\item $\tau_I = \dfrac{1}{\omega_c} \Rightarrow$ ($-45^{\circ}$ fas, men eliminerar felet för högfrekventa signaler)
\item $\tau_I = \dfrac{10}{\omega_c} \Rightarrow$ ($-5.7^{\circ}$ fas, men eliminerar inte felet för högfrekventa signaler)
\end{itemize}
Vi utförde ingen kompensering för valet av $\tau_I$ i lead-länken, så vi väljer förslaget med minst inverkan på fasen: 
\[\tau_I = \frac{10}{\omega_c}\]

\section*{Uppgift 5.10}
Betrakta systemet som är återgivet i Figur \ref{fig:5_10block}.

\begin{figure}[h!]
  \centering
  \begin{tikzpicture}[auto, node distance=2cm, >=latex']
\node [block] (g1) {$G_1(s)$};
\node [block, right of=g1, node distance=3cm] (int) {$\frac{1}{s}$};

\draw [wire] ++(-2,0) -- (g1) node[pos=0.45,above] {$u$};
\draw [wire] (g1) -- (int);
\draw [wire] (int) -- ++(2,0) node[pos=0.45, above] {$y$};
\end{tikzpicture}
  \caption{Blockschema för systemet $G(s)$}
  \label{fig:5_10block}
\end{figure}
\FloatBarrier

så att det fullständiga systemet ges av 
\[G(s) = \frac{G_1(s)}{s}\]
Bodediagrammet för $G_1(s)$ är återgivet i Figur \ref{fig:5_10bode}
\begin{figure}[h!]
  \centering
  \includegraphics[width=0.8\textwidth]{5_10bode}
  \caption{Bodediagram för $G_1(s)$}
  \label{fig:5_10bode}
\end{figure}
\FloatBarrier

Bestäm en kompensator så att
\begin{itemize}
\item $\varphi_m = 40^{\circ}$
\item Systemet är dubbelt så snabbt som ett system med $\varphi_m = 40^{\circ}$ där enbart P-reglering används.
\item Det statiska felet vid en ramp är $1\%$ av motsvarande fel för ett system med $\varphi_m = 40^{\circ}$ där enbart P-reglering används.
\end{itemize}
\qline
Bestäm först amplitud- och fasförhållandena mellan $G(s)$ och $G_1(s)$:
\begin{align*}
  |G(i\omega)| &= \frac{|G_1(i\omega)|}{\omega} \\
  \arg{G(i\omega)} &= \arg{G_1(i\omega)} - 90^{\circ}
\end{align*}
$\varphi_m = 40^{\circ}$ uppnås ifall 
\[\arg{G(i\omega_c)} = -140 ^{\circ} \Rightarrow \arg{G_1(i\omega_c)} = 50 ^{\circ}\]
Läs av ur Figur \ref{fig:5_10bode} vilket $\omega_c$ detta motsvarar: \\
$\omega_{c,d}^P = 0.52$ rad/s. \\
Bestäm ett $K_P$ så att den önskade fasmarginalen uppnås med enbart P-reglering: \\
\[K_P\cdot \frac{|G_1(i\omega_{c,d}^P)|}{\omega_{c,d}^P} = 1 \Rightarrow K_P \approx 4.2\]

Specifikationen är att systemet ska vara dubbelt så snabbt som det P-reglerade systemet. \\
$\Rightarrow \omega_{c,d} = 2\omega_{c,d}^P = 1.05$ rad/s. \\

Kolla vilken fasmarginal vi skulle ha ifall $\omega_{c,d}$ uppnås: 
\[\arg{G_1(i1.05)} = -107 ^{\circ} \Rightarrow \arg{G(i1.05)} = -197 ^{\circ}\]
$\varphi_m = 40 ^{\circ}$ kräver alltså att fasen advanceras 
\[17 ^{\circ} + 40 ^{\circ} + 6 ^{\circ} = 63 ^{\circ}\]
där $17 ^{\circ}$ flyttar fasen till $-180 ^{\circ}$, $40 ^{\circ}$ ger den önskade fasmarginalen och $6 ^{\circ}$ kompenserar för att vi kommer införa en lag-länk. Denna fasadvancering kräver två lead-länkar som höjder fasen $32 ^{\circ}$ var.
\begin{align*}
  \beta &= 0.31 \text{ Från diagram} \\
  \tau_D &= \frac{1}{\omega_{c,d}\sqrt{\beta}} = \frac{1}{1.05\sqrt{0.31}} \approx 1.72 \\ 
  |F_{lead}(i\omega_{c,d})|^2\cdot|G(i\omega_{c,d})| &= \frac{K^2}{\beta}\cdot \frac{|G_1(i\omega_{c,d})|}{\omega_{c,d}} \\
                                                     &= \frac{K^2}{0.31}\cdot \frac{0.024}{1.05} \\
                                                     &= 1 \\
  \Rightarrow K &\approx 1
\end{align*}
Introducera en lag-länk för att sänka det statiska felet vid en ramp. Notera att den rena integreringen i $G(s)$ eliminerar statiskt fel när insignalen är ett steg. Undersök det statiska felet vid ramp via slutvärdessatsen: 
\[e_1 = \lim_{s \to 0}s \cdot \frac{1}{1 + F(s) \frac{G_1(s)}{s}}\cdot \frac{A}{s^2} = \lim_{s \to 0} \frac{A}{s + F(s)G_1(s)} = \frac{A}{|F(0)|\cdot|G_1(0)|}\]
Detta gäller allmänt för alla $F(s)$. Ansätt nu att felet med lead-lag regulatorn ska vara $1\%$ av felet när P-regulatorn (med $K_P = 4.2$) används: 
\[\frac{A}{|F_{lead-lag}(0)|\cdot|G_1(0|} = 0.01\frac{A}{K_P\cdot |G_1(0)|}\]
$\Rightarrow |F_{lead-lag}(0)| = 100K_P = 420$ \\
$|F_{lead-lag}(0)| = |F_{lead}(0)|^2\cdot|F_{lag}(0)| = \dfrac{K^2}{\gamma} = 420$ \\
$\Rightarrow \gamma = \dfrac{1}{31} = 0.032$ \\

Lead-länken har kompenserat för en fasretardering med storlek $6 ^{\circ}$. Välj därför 
\[\tau_I = \frac{10}{\omega_{c,d}} \approx 10\]
Den resulterande regulatorn ges således av 
\[F_{lead-lag}(s) = 13.6 \left(\frac{1.72s+1}{0.53s+1}\right)^2 \left( \frac{10s+1}{10s+0.032}\right)\]

\section*{Uppgift 6.3}
Figur \ref{fig:6_3nyquist} visar Nyquistkurvan för ett öppet system $G_0(s)$.

\begin{figure}[h!]
  \centering
  \includegraphics[width=0.4\textwidth]{6_3nyquist}
  \caption{Nyquistkurva för ett givet $G_0(s)$}
  \label{fig:6_3nyquist}
\end{figure}
\FloatBarrier

Visa i figur för vilka frekvenser additiva störsignaler förstärks repsektive dämpas.
\qline
Det fullständiga systemet kan återges i följande blockschema \\

\begin{tikzpicture}[auto, node distance=2cm, >=latex']
\node [block] (system) {$G(s)$};
\node [block, left of=system, node distance=4cm] (reg) {$F(s)$};
\node [split, right of=system] (split) {};
\node [sum, left of=reg, node distance=3cm] (sum) {};
\node [tmp, below of=reg] (tmp){};

\draw [wire] ++(-9,0) -- node[pos=0.99] {$+$} (sum) node[pos=0.45, right, align=center, label={[laba]below:$y_{ref}(t)$}] {};
\draw [wire] (sum) -- (reg) node[pos=0.45, right, align=center, label={[laba]below:$e(t)$}] {};
\draw [wire] (system) -- ++(4,0) node[pos=0.45, right, align=center, label={[laba]below:$y(t)$}] {};
\draw [wire] (reg) -- (system) node[pos=0.45, right, align=center, label={[laba]below:$u(t)$}] {};
\draw [wire] ++(0,2) -- (system) node[pos=0.25, right, align=center, label={right:$v(t)$}] {};
\draw [wire] (split) |- (tmp) -| node[pos=0.99, right] {$-$} (sum);
\end{tikzpicture} \\

En additiv störsignal innebär att utsignalens laplacetransform får följande uttryck: 
\[Y(s) = G_c(s)Y_{ref}(s) + S(s)V(s)\]
Där $G_c(s)$ utgör det slutna systemet, $V(s)$ laplacetransformen av störsignalen, och $S(s)$ är känslighetsfunktionen som ges av
\[S(s) = \frac{1}{1 + G_0(s)}\]
Att störningen förstärks respektive dämpas innebär precis att $|S(s)| > 1$ respektive $|S(s)| < 1$. Vi bestämmer för vilka frekvenser störsignalen förstärks: 
\[|S(i\omega)| > 1 \Rightarrow |1 + G_0(i\omega)| < 1\]
Störsignalerna förstärks alltså vid de frekvenser när Nyquistkurvan ligger inom en enhetscirkel centrerad i $-1$. Situationen återgivs i Figur \ref{fig:6_3sensitivity}.

\begin{figure}[h!]
  \centering
  \includegraphics[width=0.4\textwidth]{6_3sensitivity}
  \caption{Nyquistkurva för $G_0$ med en utritad enhetscirkel som är centrerad i $-1$}
  \label{fig:6_3sensitivity}
\end{figure}
\FloatBarrier

Alltså, ifall störsignalen har frekvenser som motsvarar punkter på Nyquistkurvan inom den utritade enhetscirkeln förstärks dem under regleringen. Alla andra signalen dämpas.\\

Vi tog inte hänsyn till störningar när vi designade regulatorer i de föregående uppgifterna. Ibland måste man offra lite av prestandan som uppnås genom regulatordesign för att erhålla en snällare känslighetsfunktion.



\end{document}

