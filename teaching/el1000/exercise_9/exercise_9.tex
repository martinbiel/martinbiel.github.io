\documentclass[12pt]{article}
\usepackage[swedish]{babel}
\usepackage[T1]{fontenc}{
\usepackage[utf8]{inputenc}
\usepackage{amssymb}
\usepackage{amsmath}
\usepackage{amsfonts}
\usepackage[colorinlistoftodos,prependcaption,textsize=small]{todonotes}
\usepackage{url,graphicx,tabularx,array,geometry,color,float,placeins}
\setlength{\parskip}{1ex} %--skip lines between paragraphs
\setlength{\parindent}{0pt} %--don't indent paragraphs
%\setcounter{secnumdepth}0}
\usepackage{tikz}
\usetikzlibrary{shapes, arrows, 3d, decorations, calc, decorations.markings}

\tikzstyle{block} = [draw, rectangle, thick, minimum height=3em, minimum width=2em, align=center, text width=1.5cm, top color=white, bottom color=white!85!black]
\tikzstyle{input} = [coordinate]
\tikzstyle{output} = [coordinate]
\tikzstyle{tmp} = [coordinate]
\tikzstyle{sum} = [draw, circle, node distance=1cm]
\tikzstyle{wire} = [->,thick]
\tikzstyle{lab} = []
\tikzstyle{laba} = [lab, label distance=0cm]
\tikzstyle{labb} = [lab, label distance=0.5em]
\tikzstyle{split} = [fill,black,circle,inner sep=0.03cm]
\tikzset{cross/.style={cross out, draw=black, minimum size=2*(#1-\pgflinewidth), inner sep=0pt, outer sep=0pt},
%default radius will be 1pt. 
cross/.default={5pt}}

\newcommand{\sspan}[1]{\mathrm{span}\left\{#1\right\}}
\newcommand{\ima}[1]{\mathrm{Im}\; #1}
\newcommand{\qline}{\hrule \vspace*{10pt}}

%\linespread{2} %-- Uncomment for Double Space
\begin{document}
\begin{titlepage}
\author{Martin Biel \\ \texttt{mbiel@kth.se}}
\title{EL1000 - Reglerteknik, allmän kurs \\ \Large Övning 9}
\date{22 september 2016}
\end{titlepage}

\maketitle

\section*{Prestandabegränsningar}

\begin{itemize}
\item Modell:\\
  \begin{tikzpicture}[auto, node distance=2cm, >=latex']
\node [block] (system) {$G(s)$};
\node [block, left of=system, node distance=4cm] (reg) {$F(s)$};
\node [split, right of=system] (split) {};
\node [sum, left of=reg, node distance=3cm] (sum) {};
\node [tmp, below of=reg] (tmp){};

\draw [wire] ++(-9,0) -- node[pos=0.99] {$+$} (sum) node[pos=0.45, right, align=center, label={[laba]below:$y_{ref}(t)$}] {};
\draw [wire] (sum) -- (reg) node[pos=0.45, right, align=center, label={[laba]below:$e(t)$}] {};
\draw [wire] (system) -- ++(4,0) node[pos=0.45, right, align=center, label={[laba]below:$y(t)$}] {};
\draw [wire] (reg) -- (system) node[pos=0.45, right, align=center, label={[laba]below:$u(t)$}] {};
\draw [wire] (split) |- (tmp) -| node[pos=0.99, right] {$-$} (sum);
\end{tikzpicture}
\item Verkliga systemet:\\
\begin{tikzpicture}[auto, node distance=2cm, >=latex']
\node [block] (system) {$G^0(s)$};
\node [block, left of=system, node distance=4cm] (reg) {$F(s)$};
\node [split, right of=system] (split) {};
\node [sum, below of=split, node distance=2cm] (sum2) {};
\node [sum, left of=reg, node distance=3cm] (sum1) {};
\node [tmp, below of=reg] (tmp){};

\draw [wire] ++(-9,0) -- node[pos=0.99] {$+$} (sum) node[pos=0.45, right, align=center, label={[laba]below:$y_{ref}(t)$}] {};
\draw [wire] (sum1) -- (reg) node[pos=0.45, right, align=center, label={below:$e(t)$}] {};
\draw [wire] (system) -- ++(4,0) node[pos=0.45, right, align=center, label={above:$z(t)$}] {};
\draw [wire] (reg) -- (system) node[pos=0.45, right, align=center, label={below:$u(t)$}] {};
\draw [wire] ++(0,2) -- (system) node[pos=0.25, right, align=center, label={right:$v(t)$}] {};
\draw [wire] (split) -- node[pos=0.95, left] {$+$} (sum2);
\draw [wire] ++(4,-2) -- (sum2) node[pos=0.25, label={above:$n(t)$}] {} node[pos=0.99] {$+$};
\draw [wire] (sum2) -| node[pos=0.43,above] {$y(t)$} node[pos=0.99, right] {$-$} (sum1);
\end{tikzpicture}
\begin{itemize}
\item Störsignal: $v(t)$ \\
\item Mätfel/Mätbrus: $n(t)$ \\
\item Modellfel: $G^0(s)$ (istället för $G(s)$)
\end{itemize}
\end{itemize}
Vi har fram till nu designat våra regulator efter hypotesten att systems dynamik perfekt beskrivs av
\begin{equation}
Y(s) = G(s)U(s)\label{eq:nom}
\end{equation}

där $y(t)$ är utsignalen vi mäter. Det visar sig dock att den faktiska utsignalen vi kan mäta ges av
\begin{equation}
Y(s) = Z(s) + N(s) = G^0(s)U(s) + V(s) + N(s)\label{eq:real}
\end{equation}

Hur påverkas prestandan av att använda en regulator $F(s)$ baserad på \eqref{eq:nom}, ifall det verkliga systemet ges av \eqref{eq:real}? Vi fokuserar först på effekter från störningarna. Bestäm överföringsfunktion från referenssignal, störsignal och mätbrus till systemets faktiska uppförande $z(t)$:
\begin{align*}
  Z(s) &= G(s)U(s) + V(s) \\
  &= G(s)F(s)
    \left[
    Y_{ref}(s) - Y(s)
    \right] + V(s) \\
&= G(s)F(s)
    \left[
    Y_{ref}(s) - Z(s) - N(s)
    \right] + V(s)
\end{align*}
\[\Rightarrow Z(s) = \frac{G(s)F(s)}{1 + G(s)F(s)}Y_{ref}(s) - \frac{G(s)F(s)}{1 + G(s)F(s)}N(s) + \frac{1}{1 + G(s)F(s)}V(s) = G_c(s) - T(s)N(s) + S(s)V(s)\]
\begin{itemize}
\item $S(s)$ - Känslighetsfunktionen, förstärker störsignalen $v(t)$ 
\item $T(s)$ - Komplementära känslighetsfunktionen, förstärker mätbruset $n(t)$
\end{itemize}
Notera att $S(s) + T(s) = 1$. \\
$\Rightarrow S(s)$ och $T(s)$ kan inte vara små samtidigt. \\
$\Rightarrow$ Måste prioritera ifall störning eller mätbrus ska undertryckas.

\section*{Robusthet}
Tolerans mot modellfel.
\[G^0(s) = G(s)(1 + \Delta_G(s)\]
$\Delta_G(s) -$ Relativt modellfel (okänd, annars hade vi haft exakt vetskap om $G^0(s)$)

\textbf{Robusthetskriterium} \\
Ifall $F(s)$ stabiliserar $G(s)$, och $G(s), G^0(s)$ har samma antal poler i HHP, samt att $G(s)F(s) \to 0$ och $G^0(s)F(s) \to 0$ då $|s| \to \infty$, så gäller att slutna systemet
\[\frac{G^0(s)F(s)}{1 + G^0(s)F(s)}\]
är stabilt \emph{om}
\begin{equation}
|\Delta_G(i\omega) < \frac{1}{|T(i\omega)|} \forall \omega\label{eq:robust}
\end{equation}

(\emph{Tillräckligt} men \emph{inte nödvändigt} villkor, systemet kan vara stabilt även om inte \eqref{eq:robust})

\section*{Uppgift 6.7}



\end{document}

