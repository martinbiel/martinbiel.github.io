\documentclass[12pt]{article}
\usepackage[swedish]{babel}
\usepackage[T1]{fontenc}{
\usepackage[utf8]{inputenc}
\usepackage{amssymb}
\usepackage{amsmath}
\usepackage{amsfonts}
\usepackage[colorinlistoftodos,prependcaption,textsize=small]{todonotes}
\usepackage{url,graphicx,tabularx,array,geometry,color,float,placeins}
\setlength{\parskip}{1ex} %--skip lines between paragraphs
\setlength{\parindent}{0pt} %--don't indent paragraphs
%\setcounter{secnumdepth}0}
\usepackage{tikz}
\usetikzlibrary{shapes, arrows, 3d, decorations}

\tikzstyle{block} = [draw, rectangle, thick, minimum height=3em, minimum width=2em, align=center, text width=1.5cm, top color=white, bottom color=white!85!black]
\tikzstyle{input} = [coordinate]
\tikzstyle{output} = [coordinate]
\tikzstyle{tmp} = [coordinate]
\tikzstyle{sum} = [draw, circle, node distance=1cm]
\tikzstyle{wire} = [->,thick]
\tikzstyle{lab} = []
\tikzstyle{laba} = [lab, label distance=0cm]
\tikzstyle{labb} = [lab, label distance=0.5em]
\tikzstyle{split} = [fill,black,circle,inner sep=0.03cm]

\newcommand{\sspan}[1]{\mathrm{span}\left\{#1\right\}}
\newcommand{\ima}[1]{\mathrm{Im}\; #1}
\newcommand{\qline}{\hrule \vspace*{10pt}}

%\linespread{2} %-- Uncomment for Double Space
\begin{document}
\begin{titlepage}
\author{Martin Biel \\ \texttt{mbiel@kth.se}}
\title{EL1000 - Reglerteknik, allmän kurs \\ \Large Övning 3}
\date{31 augusti 2016}
\end{titlepage}

\maketitle

\section*{Repetition}

\begin{itemize}
\item \textbf{Återkopplat system} \\
\begin{tikzpicture}[auto, node distance=2cm, >=latex']
\node [block] (system) {$G(s)$};
\node [block, left of=system, node distance=4cm] (reg) {$F(s)$};
\node [split, right of=system] (split) {};
\node [sum, left of=reg, node distance=3cm] (sum) {};
\node [tmp, below of=reg] (tmp){};

\draw [wire] ++(-9,0) -- node[pos=0.99] {$+$} (sum) node[pos=0.45, right, align=center, label={[laba]below:$y_{ref}(t)$}] {};
\draw [wire] (sum) -- (reg) node[pos=0.45, right, align=center, label={[laba]below:$e(t)$}] {};
\draw [wire] (system) -- ++(4,0) node[pos=0.45, right, align=center, label={[laba]below:$y(t)$}] {};
\draw [wire] (reg) -- (system) node[pos=0.45, right, align=center, label={[laba]below:$u(t)$}] {};
\draw [wire] ++(0,2) -- (system) node[pos=0.25, right, align=center, label={right:$v(t)$}] {};
\draw [wire] (split) |- (tmp) -| node[pos=0.99, right] {$-$} (sum);
\end{tikzpicture}

\textbf{Reglerfelet:} $E(s) = Y(s) - Y_{ref}(s)$ \\
\textbf{Regulator:} $F(s)$ \\

\item \textbf{Öppna systemet:} $G_0 (s) = F(s)G(s)$
\item \textbf{Slutna systemet:} $G_c (s) = \dfrac{G_0 (s)}{1 + G_0 (s)}$
\item \textbf{PID-Regulator:} $F_{PID}(s) = K_P + \dfrac{K_I}{s} + K_D s$
\end{itemize}

\section*{Andra ordningens system}

\begin{equation*}
  G(s) = \frac{b}{s^2 + a_1 s + a_2}
\end{equation*}
Inför: \\
\begin{align*}
  \omega_0 &= \sqrt{a_2} \text{ - Egenfrekvens} \\
  \zeta &= \frac{a_1}{2\sqrt{a_2}} \text{ - Relativ dämpning} \\
  K &= \frac{b}{a_2} \text{ - förstärkning}
\end{align*}
\begin{equation*}
  \Rightarrow G(s) = \frac{K\omega_0^2}{s^2 + 2\zeta \omega_0s + \omega_0^2}
\end{equation*}
Identifiera systemets poler
\begin{equation*}
  s^* = -\zeta\omega_0 \pm \omega\sqrt{\zeta^2-1} = -\zeta\omega_0 \pm i \omega\sqrt{1 - \zeta^2}
\end{equation*}
Notera att vi får komplexkonjugerade poler då $\zeta < 1$. \\
Kom ihåg att polernas avstånd från origo påverkar systemets hastighet och att storleken på imaginärdelen påverkar systemets svängighet/dämpning. \\

Identifiera belopp och argument: \\
\begin{align*}
|s^*| &= \sqrt{(\zeta\omega)^2 + \omega^2(1-\zeta^")} = \omega_0 \\
\arg{s^*} &= \phi \text{, där } \cos{\phi} = \frac{\zeta\omega}{\omega} = \zeta 
\end{align*}
$\omega_0$ relaterar således till systemets hastighet då dess värde avgör polernas avstånd från origo. Kortfattat gäller:
\begin{itemize}
\item $\omega_0$ stor - snabbt system
\item $\omega_0$ liten - långsamt system
\end{itemize}
$\zeta$ relaterar till systemets dämpning då dess värde avgör hur stor imaginärdel polerna får. Kortfattat gäller (för $\zeta < 1$:
\begin{itemize}
\item $\zeta$ stor - svagt dämpat systemet (svängigt)
\item $\zeta$ liten - start dämpat system (inte svängigt)
\end{itemize}
För $\zeta > 1$ erhålls ett starkt dämpat system som beter sig likt ett första ordningens system (fast med två exponentialtermer). För $\zeta = 1$ erhålls ett kritiskt dämpat system med överföringsfunktion
\[G(s) = \frac{K}{(\frac{s}{w_0} + 1)^2}\]
Dubbelpolen i $-\omega_0$ ger för $\omega_0 > 0$ upphov till ett system som rör sig tillbaka till jämvikt fortast möjligt för ett andra ordningens system.

\section*{Uppgift 3.2}
Givet från uppgift 3.1 är det slutna systemet
\begin{equation*}
  H(s) = \frac{2F(s)}{s(1+5s)+2F(s)} H_{ref}(s)
\end{equation*}
\subsection*{a)}
Vilka poler får systemet ifall $F(s)$ är en P-regulator med $K_P = 1$?
\qline
Nämnarpolynomet för det slutna systemet ges av
\[s(1+5s) + 2F(s) = 5s^2 + s + 2\]
Nollställena till detta polynom ges av
\[s^* = -0.1 \pm 0.62i\]
Stegsvaret för det resulterande systemet är återgivet i Figur \ref{fig:32stepP}.
\begin{figure}[h!]
  \centering
  \includegraphics[width=0.5\textwidth]{3_2stepP}
  \caption{Stegsvar för systemet då P-reglering används}
\label{fig:32stepP}
\end{figure}
\FloatBarrier
Notera att det inte förekommer något stationärt fel, även fast vi inte inkluderande en integrator i $F(s)$. Ifall det nominella systemet $G(s)$ (utan reglering) inehåller en ren integrering så erhålls inget stationärt fel för det slutna systemet. Det kvittar alltså ifall integratorn är en del av regulatorn eller det givna systemet. För att se varför, undersök hur referenssignalen $H_{ref}(s)$ överförs till felet $E(s)$: \\
\begin{align*}
  E(s) &= H_{ref}(s) - H(s) = H_{ref}(s) - G(s)U(s) = H_{ref}(s) - F(s)G(s)E(s) \\
  \Rightarrow E(s) &= \frac{1}{1+F(s)G(s)}H_{ref}(s)
\end{align*}
Använd slutvärdessatsen då $H_{ref}(s)$ är ett stegsvar:
\begin{equation*}
  \lim_{t \to \infty} e(t) = \lim_{s \to 0}s \cdot \frac{1}{1 + F(s)G(s)} \cdot \frac{1}{s} = \lim_{s \to 0} \frac{1}{1 + F(s)G(s)}
\end{equation*}
Härav följar att ifall antingen $F(s) \to \infty$ eller $G(s) \to \infty$ då $s \to 0$ så kommer $e(t) \to 0$ då $t \to \infty$. Således räcker det med att en ren integrering förekommer i antingen $F(s)$ eller $G(s)$ för att det stationära felet ska gå mot noll.

\subsection*{b)}
Använd en PD-regulator så att den relativa dämpningen blir större än $\frac{1}{\sqrt{2}}$.
\qline
Nu är $F(s) = K_P + K_D s$. Nämnarpolynomet ges nu istället av
\begin{equation*}
  s(1+5s)+2(K_P + K_D s) = s^2 + \frac{1+2K_D}{5}s + \frac{2K_P}{5}
\end{equation*}
Identifiera egenfrekvensen och den relativa dämpningen ($s^2 + 2\zeta\omega_0 s + \omega_0^2$):
\begin{align*}
  \omega_0 &= \sqrt{\frac{2K_P}{5}} \\
  \zeta &= \frac{\frac{1+K_D}{5}}{2\sqrt{\frac{2K_P}{5}}} = \frac{1+2K_D}{2\sqrt{10K_P}}
\end{align*}
Notera att $K_D$ inte påverkar $\omega_0$. Det är således möjligt att först fixera $K_P$ så att en önskad snabbhet uppnås och sedan välja $K_D$ för att reducera överslängen i stegsvaret. Antag att $K_P = 1$, vilket här bara är ett godtyckligt val.
\begin{equation*}
  \frac{1+2K_D}{2\sqrt{10}} > \frac{1}{\sqrt{2}} \Leftrightarrow K_D > \frac{\sqrt{20} - 1}{2} \approx 1.7
\end{equation*}
Med de valda parameterna $K_P = 1$ och $K_D = 1.7$ erhålls stegsvaret i Figur \ref{fig:32stepPD}.

\begin{figure}[h!]
  \centering
  \includegraphics[width=0.5\textwidth]{3_2stepPD}
  \caption{Stegsvar för systemet då PD-reglering används}
  \label{fig:32stepPD}
\end{figure}
Jämfört med Figur \ref{fig:32stepP} är det tydligt att systemet nu blivit mer dämpat.

\section*{Rotort}
Diagram över det \emph{slutna} systemets poler som funktion av någon parameter $K$ (ofta gäller $0 \leq K \leq \infty$.

\textbf{Rita rotort:}
\begin{enumerate}
\item Ta fram $G_c(s)$
\item Skriv nämnarpolynomet på formen
\[P(s) + KQ(s)\]
Där $n = \mathrm{grad}{P} \geq \mathrm{grad}{Q} = m$
\item Hitta startpunkter till rotorten
\[K = 0 \Leftrightarrow P(s) = 0\]
Ges av nollställen till $P(s)$ (finns $n$ stycken)
\item Hitta ändpunkter till rotorten
\[K \to \infty \Leftrightarrow Q(s) \to 0\]
Motsvarar nollställena till $Q(s)$ (finns $m$ stycken)
\item Bestäm antal asymptoter \\
Ges av $n-m$
\item Bestäm skärningspunkt (för asympoterna) \\
Ges av 
\[\frac{1}{n-m}\left(\sum \text{Startpunkter} - \sum \text{Ändpunkter}\right)\]
Notera at startpunkterna och ändpunkternas \emph{värden} summeras, inte deras antal.
\item Hitta asymptoternas riktningar \\
Ges av
\[\frac{\pi}{n-m} + 2k\frac{\pi}{n-m}, k = 0,1,2,\cdots. n-m-1\]
\item Bestäm eventuell skärning med Im-axeln \\
Motsvarar att lösa $P(i\omega) + KQ(i\omega) = 0$
\item Bestäm de delar av Re-axeln som tillhör Rotorten. Minnesregel är att punkter på Re-axeln med ett udda antal start- och ändpunkter till höger tillhör rotorten.
\item Rita rotort
\end{enumerate}
En kommentar till punkt 9: \\
Vid ett nollställe till nämnarpolynomet gäller $P(s) + KQ(s) = 0$ vilket är ekvivalent med
\[\frac{P(s)}{Q(s)} = -K\]
och då vi antar att $K > 0$ implicerar uttrycket ovan att kvoten mellan $P(s)$ och $Q(s)$ måste vara negativ för att tillhöra rotorten. Ifall polynomen faktoriseras 
\[\frac{P(s)}{Q(s)} = \frac{(s-p_1)(s-p_2)\cdots(s^2+\alpha_1s + \alpha_2)\cdots}{(s-q_1)(s-q_2)\cdots(s^2+\gamma_1s+\gamma_2)\cdots} \]
där komplexkonjugerande nollställen ger upphov till andragradsfaktorer. Det är nu tydligt att 
\[\frac{P(s)}{Q(s)}\]
är positiv för $s >> 0$ och byter inte tecken då $s$ minskar förrän en startpunkt ($s = p_i$) eller ändpunkt ($s = q_i$) passeras. Således kommer kvoten vara negativ för alla punkter på Re-axeln som ligger till vänster av ett udda antal start- eller ändpunkter.\\\\
Notera att rotorten är entydig för varje $K$. En uppsättning poler motsvarar ett unikt $K$. Således kan en specifik punkt bara passeras en gång då rotorten ritas.

\section*{Uppgift 3.6}

\begin{tikzpicture}[auto, node distance=2cm, >=latex']
\node [block] (int) {$\dfrac{1}{s}$};
\node [block, left of=int, node distance=4cm] (dc) {$\dfrac{\kappa}{1+s\tau}$};
\node [block, left of=dc, node distance=3cm] (K) {$K$};
\node [block, below of=K, node distance=1.5cm, xshift=1.5cm] (a) {$\alpha$};
\node [split, left of=int] (split1) {};
\node [split, right of=int] (split2) {};
\node [sum, left of=K, node distance=3cm] (sum1) {};
\node [sum, left of=K, node distance=1.5cm] (sum2) {};
\node [tmp, below of=reg, node distance=2.5cm] (tmp){};

\draw [wire] ++(-12,0) -- node[pos=0.99] {$+$} (sum1) node[pos=0.45, right, align=center, label={[laba]above:$\theta_{ref}(t)$}] {};
\draw [wire] (sum1) -- node[pos=0.99] {$+$} (sum2);
\draw [wire] (sum2) -- (K);
\draw [wire] (int) -- ++(4,0) node[pos=0.45, right, align=center, label={[laba]above:$\theta(t)$}] {};
\draw [wire] (K) -- (dc);
\draw [wire] (dc) -- (int) node[pos=0.45, right, align=center, label={[laba]above:$\dot{\theta}(t)$}] {};
\draw [wire] (split1) |- (a) -| node[pos=0.99, right] {$-$} (sum2);
\draw [wire] (split2) |- (tmp) -| node[pos=0.99, right] {$-$} (sum1);
\end{tikzpicture}
$\tau = 0.5$ \\
$\kappa = 2$ \\
\newpage
Rita rotort med avseende på $K$ för $\alpha = 0$
\qline
Följ proceduren ovan.\\
\textbf{1)} \\
Gå i blockdiagrammet för att bestämma $G_c(s)$
\begin{align*}
  \Theta(s) &= \frac{1}{s}\dot{\Theta}(s) \\
  \dot{\Theta}(s) &= \frac{2K}{1+0.5s}U(s) \\
  U(s) &= E(s) - \alpha\dot{\Theta}(s) \\
  E(s) &= \Theta_{ref}(s) - \Theta (s) \\
  \Rightarrow \dot{\Theta}(s) &= \frac{2K}{1+0.5s}\left(\Theta_{ref}(s) - \Theta(s) - \alpha\dot{\Theta}(s)\right) \\
  \Rightarrow \dot{\Theta}(s) &= \frac{\frac{2K}{1+0.5s}}{1 + \frac{2\alpha K}{1+0.5s}}\left(\Theta_{ref}(s) - \Theta(s)\right) = \frac{2K}{1 + 0.5s + 2\alpha K}\left(\Theta_{ref}(s) - \Theta(s)\right) \\
  \Rightarrow \Theta(s) &= \frac{2K}{s(1+0.5s+2\alpha K)}\left(\Theta_{ref}(s) - \Theta(s)\right) \\
  \Rightarrow \Theta(s) &= \frac{\frac{2K}{s(1+0.5s+2\alpha K)}}{1 + \frac{2K}{s(1+0.5s+2\alpha K)}}\left(\Theta_{ref}(s) - \Theta(s)\right) = \frac{2K}{0.5s^2 + s + (2\alpha + 2)K}\Theta_{ref}(s)
\end{align*}
Överföringsfunktionen för slutna systemet, då $\alpha = 0$ ges således av
\[\frac{2K}{0.5s^2 + s + 2K}\]
\textbf{2)} \\
\begin{align*}
  P(s) &= 0.5s^2 + s (n = 2) \\
  Q(s) &= 2 (m = 0)
\end{align*}
\textbf{3)} \\
\begin{align*}
  P(s) &= 0 \\
  \Leftrightarrow s(0.5s+1) &= 0 \\
  \Rightarrow s &= 0, s = -2
\end{align*}
Rotorten startar alltså i polerna $s = 0$ och $s = -2$. \\
\textbf{4)} Rotorten saknar ändpunkter \\
\textbf{5)} Antalet asymptoter ges av $n-m = 2$. \\
\textbf{6)} Asymptoternas skärningspunkt ges av
\[\frac{1}{2}(-2-0) = -1\]
\textbf{7)} Asymptoternas riktningar ges av $\dfrac{\pi}{2}$, $\dfrac{3\pi}{2}$. \\
\textbf{8)} 
\begin{align*}
  0.5(i\omega)^2 + i\omega + 2K &= 0 \\
  \Leftrightarrow (-0.5\omega^2 + 2K) + i\omega &= 0
\end{align*}
Im-del: $\omega = 0$ \\
Re-del $\omega = \pm 2\sqrt{K}$ \\
$\Rightarrow$ enda lösningen är $\omega = 0$ då $K = 0$! Detta är ekvivalent med att rotorten skär imaginär axeln i origo. \\
\textbf{9)} Alla punkter mellan $-2$ och $0$ tillhör rotorten då de ligger till vänster om \emph{en} startpunkt. Inga andra delar av Re-axeln tillhör rotorten\\\\
Eftersom rotorten är entydig tvingas polerna mot varandra (från $-2$ och $0$) tills de möts på mitten i $-1$. Där tvingas polerna rakt ut (Riktningar: $90^{\circ}$ och $270^{\circ}$)  i komplexa talplanet. Rotorten visas i Figur \ref{fig:36rlocus}. I detta triviala fall kan polerna beräknas explicit som funktion av $K$:
\[s^* = -1 \pm \sqrt{1-4K}\]
Lösningen ger intuition för rotortens utseende. För $K < 1/4$ erhålls reella poler mellan $-2$ och $0$. Då $K > 1/4$ erhålls komplex konjugerande punkter som utgår från skärningspunkten $-1$ och rör rakt ut i komplexa talplanet (en pol rakt upp, den andra rakt ner) för ökande $K$.
\begin{figure}[h!]
  \centering
  \includegraphics[width=0.8\textwidth]{3_6rlocus}
  \caption{Rotort med avseende på $K$ då $\alpha = 0$.}
  \label{fig:36rlocus}
\end{figure}
\FloatBarrier

\section*{Uppgift 3.7}

Vi betraktar en luftfarkost som flyger med konstant pitch $\theta$, se Figur \ref{fig:37plane}. \\
\begin{figure}[h!]
  \centering
  \includegraphics[width=0.8\textwidth]{3_7plane}
  \caption{Flygfarkost}
  \label{fig:37plane}
\end{figure}
\FloatBarrier
Låt
\[\omega(t) = \dot{\theta}(t)\]
Nära ett givet jämviktsläge $\theta_0$ gäller
\[\Omega (s) = \frac{s+1}{(s+4)(s-3)}\Delta (s) := G_2(s)\Delta(s)\]
Rodret drivs via
\[\Delta (s) = \frac{10}{s + 10}\Delta_{ref}(s) := G_1(s) \Delta_{ref}(s)\]
Alltså har vi följande överföringsfunktion från efterfrågat roderutslag till pitch-hastighet:
\[\Omega(s) = G_2(s)G_1(s)\Delta_{ref}(s) = \frac{10(s+1)}{(s+10)(s+4)(s-3)}\Delta_{ref}(s)\]

\subsection*{a)}
Vad händer med $\omega$ ifall vi begär ett konstant roderutslag $\delta_{ref}$?
\qline
Det nominella systemet är instabilt (pol i $s = 3$). Således gäller att $\omega(t) \to \infty$ ifall vi applicerar en konstant stegsignal.

\subsection*{b)}
$\omega$ mäts och används i en negativ återkoppling för att styra systemet med en P-regulator ($\Delta_{ref} = K(\Omega_{ref}(s) - \Omega(s))$. Rita rotort för det slutna systemet. För vilka $K$ är det slutna systemet stabilt?
\qline
Som tidigare följer vi proceduren steg för steg. \\
\textbf{1)}
\begin{align*}
  \Omega (s) &= G_2(s)G_1(s)K(\Omega_{ref}(s) - \Omega(s)) \\
  \Rightarrow \Omega(s) &= \frac{KG_2(s)G_1(s)}{1 + KG_2(s)G_1(s)}\Omega_{ref}(s) = \frac{10K(s+1)}{(s+10)(s+4)(s-3) + 10K(s+1)}\Omega_{ref}(s) \\
\end{align*}
\textbf{2)}
\begin{align*}
  P(s) &= (s+10)(s+4)(s-3) (n = 3) \\
  Q(s) &= 10(s+1) (m = 1)
\end{align*}
\textbf{3)}
\begin{align*}
  P(s) &= 0 \\
  \Rightarrow s &= -10, s = -4, s = 3
\end{align*}
Rotorten startar alltså i polerna $s = -10$, $s = -4$ och $s = 3$. \\
\textbf{4)}
\begin{align*}
  Q(s) &= 0 \\
  \Rightarrow s &= -1
\end{align*}
Rotorten slutar alltså i polen $s = -1$ \\
\textbf{5)} Antalet asymptoter ges av $3 - 1 = 2$ \\
\textbf{6)} Skärningspunkten ges av
\[\frac{1}{3-1}(-10-4+3 - (-1)) = -5\]
\textbf{7)} Asymptoternas riktningar ges av $\frac{\pi}{2}$, $\frac{3\pi}{2}$. \\
\textbf{8)} 
\begin{align*}
  P(i\omega) + KQ(i\omega) &= 0 \\
  \Leftrightarrow (i\omega+10)(i\omega+4)(i\omega-3) + 10K(i\omega+1) &= 0 \\
  \Leftrightarrow (-\omega^2+14i\omega+40)(i\omega-3) + 10K(i\omega+1) &= 0 \\
  \Leftrightarrow -i\omega^3+3\omega^2-14i\omega-42i\omega + 40i\omega-120 + 10K(i\omega+1) &= 0 \\
  \Leftrightarrow (-\omega^2 + 10K - 2)\omega i + (-11\omega^2 + 10K - 120) &= 0
\end{align*}
Im-del: $-(\omega^2-10K+2)\omega = 0$ \\
Re-del: $11\omega^2 - 10K + 120 = 0$ \\
Im-delen har lösningarna
\[\omega = 0, \omega = \sqrt{10K-2}\]
Re-delen har lösningen (för $\omega \neq 0$)
\[\omega = \frac{\sqrt{10K-120}}{11} \neq \sqrt{10K-2}\]
$\Rightarrow$ enda lösningen är $\omega = 0$ vilket ger (insättning $\omega = 0$ i Re-delen)
\[K = \frac{120}{10} = 12\]
\textbf{9)} \\
Punkter mellan $-1$ och $3$ tillhör rotorten (vänster om en startpunkt). \\
Punkter mellan $-10$ och $-4$ tillhör rotorten (vänster om två startpunkter och en ändpunkt) \\\\
Polen som börjar i $s = 3$ slutar eventuellt i ändpunkten $s = -1$ när $K \to \infty$. Polerna som börjar i $s = -10$ och $s = -4$ tvingas mot varandra då $K$ växer, och tvingas ut i komplexa talplanet då de möts i mitten (i punkten $s = -7$). Då $K \to \infty$ rör sig det komplexa polparet ut mot asymptoterna (som har skärningspunkt $5$). Den resulterande rotorten visas i Figur \ref{fig:37rlocus}
\begin{figure}[h!]
  \centering
  \includegraphics[width=0.8\textwidth]{3_7rlocus}
  \caption{Rotort med avseende på $K$}
  \label{fig:37rlocus}
\end{figure}
Skärningen med Im-axeln sker när $K = 12$. Polerna som eventuellt rör sig ut i det komplexa talplanet har negativ realdel för samtilga $K$. Således befinner sig alla poler i VHP för $K > 12$ vilket därför blir vårat stabilitetskriterium. \\
\textbf{Svar:} Systemet är stabilt när $K > 12$.

\subsection*{c)}
Finns det något värde på $K$ där systemet är stabilt och alla poler befinner sig på realaxeln?\vspace*{2pt}
\qline
Beräkna först det $K$ där två poler börjar tvingas ut i det komplexa talplanet, vilket motsvarar punkten $s = -7$. Insättning ger:
\begin{align*}
  P(-7) + KQ(-7) &= 0 \\
  \Leftrightarrow (-7+10)(-7+4)(-7-3) + 10K(-7+1) &= 0 \\
  \Leftrightarrow 9-&K &= 0 \\
  \Rightarrow K &= \frac{3}{2} = 1.5
\end{align*}
Således får det slutna systemet komplexkonjugerande poler då $K > 1.5$. Eftersom systemet blir stabilt först för $K > 12$ är det omöjligt att erhålla ett stabilt system med realvärda poler.\\
\textbf{Svar:} Det finns inget $K$ för vilket systemet är stabilt och har enbart realvärda poler. \\\\
Den sista uppgiften exemplifierar en begränsning med P-regulatorn. Det är ibland omöjligt att erhålla ett stabilt slutet system utan att även införa oscillationer i utsignalen. Med hjälp av rotorten kan vi identifiera hur det $K$ värde som krävs för att erhålla samtliga poler i VHP (för stabilitet) även tvingar ut vissa poler i det komplexa talplanet (vilket ger upphov till svängningar).

\end{document}
