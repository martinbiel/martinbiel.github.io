\documentclass[12pt]{article}
\usepackage[swedish]{babel}
\usepackage[T1]{fontenc}{
\usepackage[utf8]{inputenc}
\usepackage{amssymb}
\usepackage{amsmath}
\usepackage{amsfonts}
\usepackage[colorinlistoftodos,prependcaption,textsize=small]{todonotes}
\usepackage{url,graphicx,tabularx,array,geometry,color,float,placeins}
\setlength{\parskip}{1ex} %--skip lines between paragraphs
\setlength{\parindent}{0pt} %--don't indent paragraphs
%\setcounter{secnumdepth}0}
\usepackage{tikz}
\usetikzlibrary{shapes, arrows, 3d, decorations, calc, decorations.markings}

\tikzstyle{block} = [draw, rectangle, thick, minimum height=3em, minimum width=2em, align=center, text width=1.5cm, top color=white, bottom color=white!85!black]
\tikzstyle{input} = [coordinate]
\tikzstyle{output} = [coordinate]
\tikzstyle{tmp} = [coordinate]
\tikzstyle{sum} = [draw, circle, node distance=1cm]
\tikzstyle{wire} = [->,thick]
\tikzstyle{lab} = []
\tikzstyle{laba} = [lab, label distance=0cm]
\tikzstyle{labb} = [lab, label distance=0.5em]
\tikzstyle{split} = [fill,black,circle,inner sep=0.03cm]
\tikzset{cross/.style={cross out, draw=black, minimum size=2*(#1-\pgflinewidth), inner sep=0pt, outer sep=0pt},
%default radius will be 1pt. 
cross/.default={5pt}}

\newcommand{\sspan}[1]{\mathrm{span}\left\{#1\right\}}
\newcommand{\ima}[1]{\mathrm{Im}\; #1}
\newcommand{\qline}{\hrule \vspace*{10pt}}
\newcommand{\de}[1]{\mathrm{d}#1}

%\linespread{2} %-- Uncomment for Double Space
\begin{document}
\begin{titlepage}
\author{Martin Biel \\ \texttt{mbiel@kth.se}}
\title{EL1000 - Reglerteknik, allmän kurs \\ \Large Övning 6}
\date{14 september 2016}
\end{titlepage}

\maketitle

\section*{Repetition}
\textbf{Linjärt tids-invariant system (LTI)}:
\begin{itemize}
\item $\alpha_1 u_1(t) + \alpha_2u_2(t) \rightarrow \alpha_1 y_1(t) + \alpha_2 y_2(t)$ 
\item $u(t-T) \rightarrow y(t-T), \forall T$
\end{itemize}
$\Rightarrow$ En summa av olika insignaler superpositioneras och det spelar ingen roll \emph{när} vi lägger på insignalen, utan \emph{hur länge} den har verkat. \\

\section*{Frekvenssvar}
Låt $u(t) = \sin{\omega t} = \ima{e^{i\omega T}}$ och minns att utsignalen kan bestämmas genom en faltning mellan insignalen $u(t)$ och överföringsfunktionen 
\[G(s) = \int_{0}^{\infty}g(\tau) e^{-s \tau} \de{\tau}
\]
så att
\begin{align*}
  y(t) &= \int_{0}^{\infty}g(\tau) u(t-\tau) \de{\tau} \\
       &= \ima{\int_{0}^{\infty}e^{i\omega(t-\tau)}\de{\tau}} \\
       &= \ima{\int_{0}^{\infty}g(\tau)e^{-i\omega \tau}\de{\tau} \cdot e^{i\omega t}} \\
       &= \ima{G(i\omega) \cdot e^{i\omega t}} \\
       &= \ima{|G(i\omega)| \cdot e^{i\arg{G(i\omega)}} \cdot e^{i\omega t}} \\
       &= |G(i\omega)| \sin{\omega t + \phi}
\end{align*}
där $\phi = \arg{G(i\omega)}$. \\
$\Rightarrow$ Insignalen förstärks med $|G(i\omega)|$ och förskjuts med $\arg{G(i\omega)}$. \\

Eftersom linjära system superpositionerar insignaler, och godtyckliga funktioner kan skrivas som en serie trigonometriska funktioner (Fourierserie), så är det tillräckligt att analysera $G(i\omega)$ för att dra slutsatser om hur systemet påverkas av olika insignaler. $G(i\omega)$ kallas systemets \emph{frekvenssvar}.

\section*{Bodediagram}
Bodediagram består av två diagram
\begin{itemize}
\item $|G(i\omega)|$ - Beloppskurva (Ofta i logaritmisk skala) 
\item $\arg{G(i\omega)}$ - Faskurvan
\end{itemize}
Nyquistkurvan är precis 
\[G(i\omega), \omega: 0 \to \infty\]
Alltså visar Bodediagrammen och Nyquistkurvan samma sak! Deras relation är illustrerad i Figur

\begin{figure}[h!]
  \centering
  \includegraphics[width=0.8\textwidth]{bodenyquist}
  \caption{Samband mellan Bodediagrammet och Nyquistkurvan (Källa: Mariette Annergren}
  \label{fig:bodenyquist}
\end{figure}
\FloatBarrier

I Bodediagrammet (och även indirekt i Nyquistkurvan) kan några storheter som ger mycket information om systemet definieras:

\begin{itemize}
\item $\omega_p$ - Fasskärfrekvens \\
\begin{itemize}
 \item Bode: Faskurvan skär $-180 ^{\circ}$ 
 \item Nyquist: kurvan skär negativa Re-axeln
\end{itemize}
\item $\omega_c$ - Skärfrekvens \\
\begin{itemize}
\item Bode: Beloppskurvan skär $1$ ($0$ dB) 
\item Nyquist: Kurvan skär enhetscirkeln
\end{itemize}
\item $\varphi_m$ - Fasmarginal
\begin{itemize}
\item Bode: Vinkelavståndet mellan fasen vid $\omega_c$ och $-180 ^{\circ}$
\item Nyquist: Vinkeln mellan negativa Re-axeln och skärningen med enhetscirkeln
\end{itemize}
\item $A_m$ - Amplitudmarginal
\begin{itemize}
\item Bode: Logaritmiskt avstånd mellan beloppet vid $\omega_p$ och $0$ dB (belopp $1$)
\item Nyquist: Inversa avståndet mellan origo och skärningen med negativa Re-axeln
\end{itemize}



\end{itemize}




\subsection*{Elementära Bodediagram}
Betrakta 
\[G(s) = \frac{b}{s+a}\]
för små frekvenser gäller 
\[G(s)_{lf} = \frac{b}{a}\]
en konstant, och för stora frekvenser gäller 
\[G(s)_{hf} = \frac{b}{s}\]
Det är nu möjligt att identifiera en brytpunkt när asymptoterna skär varandra: 
\[\frac{b}{a} = \frac{b}{s} \Rightarrow s = a\]
när bodediagrammet når brytpunkten $s = a$ sker en negativ lutningsförändring. På samma sätt gäller att när bodediagrammet för någon annan $\tilde{G}(s)$, med nollställen, når en brytpunkt i täljaren så sker en positiv lutningsförändring. Dessa betraktelser går att generalisera till allmänna bodediagram.
\subsection*{Skissa Bodediagram}

\begin{enumerate}
\item Faktorisera $G(s)$:
  \[G(s) = \frac{K(1+s/z_1)(1+s/z_2)\dots(1+s/z_m)}{s^p(1+s/p_1)(1+s/p_2)\dots(1+s/p_n)}\]
  Nu gäller 
  \[\log{|G(i\omega)|} = \log{K} - p\log{\omega} + \log{\left|1+\frac{i\omega}{z_1}\right|} + \dots + \log{\left| 1 + \frac{i\omega}{z_m}\right|} - \log{\left|1 + \frac{i\omega}{p_1}\right|} - \dots - \log{\left| 1 + \frac{i\omega}{p_m}\right|}\] 
  \[\arg{G(i\omega)} = -\frac{\pi p}{2} + \arg{1 + \frac{i\omega}{z_1}} + \dots + \arg{1+\frac{i\omega}{z_m}} - \arg{1 + \frac{i\omega}{p_1}} - \dots - \arg{1 + \frac{i\omega}{p_n}}\]
Bodediagrammet är alltså en summa av elementära Bodediagram!
\item Beräkna lågfrekvensasymptot ( $\omega \to 0$ ) 
\item Beräkna högfrekvensasymptot ( $\omega \to \infty$ )
\item Identifiera brytpunkter: $z_1, z_2, \dots, z_m, p_1, p_2, \dots, p_n$
\item Identifiera bidrag till lutning (följer från de elementära Bodediagrammen):
\begin{itemize}
\item Brytpunkt från en pol $\rightarrow -1$ dekad/dekad i lutning
\item Brytpunkt från nollställe $\rightarrow 1$ dekad/dekad i lutning
\item Inledningsvis är lutningen $-p$
\end{itemize}
\item Förankra Bodediagrammet i en godtycklig punkt $\bar{\omega}$ genom att beräkna $|G(i \bar{\omega})|$
\item Beräkna $\arg{G(i\omega)}$ för några olika $\omega$ och interpolera faskurvan.
\end{enumerate}

\section*{Uppgift 4.2}
Målet är att behålla en given kurs $\Phi$ för ett fartyg, låt 
\[\omega = \dot{\Phi}\]
För små $\omega$ och $\delta$ gäller 
\[T_1 \dot{\omega} = -\omega+k_1 \delta\]
där $T_1 = 100$ och $k_1 = 0.1$. Fartyget är utrustat med en autopilot 
\[F(s) = K \frac{1+s/a}{1+s/b}\]
(där $a = 0.02$ och $b = 0.05$) vars mål är att se till att fartyget bibehåller kursen. Den beräknade insignalen styr ett roder med följande dynamik 
\[G_r(s) = \frac{1}{1+sT_2}\]
där $T_2 = 10$. Totalt gäller alltså 
\[\Phi(s) = G_s(s)G_r(s)F(s)\]
där
$\Phi(s) = G_s(s) \Delta(s)$

\subsection*{a)}
Rita Bodediagram för $FG_rG_s$ för $K = 0.5$
\qline
Bestäm först $G_s(s)$:
\begin{align*}
  T_1 \ddot{\Phi} + \dot{\Phi} &= k_1 \delta \\
  \Rightarrow s^2 T_1 \Phi(S) + s \Phi(s) &= k_1 \Delta(s) \\
  \Rightarrow \Phi(s) &= \frac{k_1}{T_1s^2+s}\Delta(s)
\end{align*}
så att 
\[G(s) = \frac{k_1}{T_1 S^2 +s}\]
Alltså gäller 
\[G_0(s) = F(s)G_r(s)G_s(s) = K\frac{1+\frac{s}{a}}{1+\frac{s}{b}} \cdot \frac{1}{1+sT_2} \cdot \frac{k_1}{T_1 s^2+s}\]

1) Faktorisera:
\[G_0(s) = \frac{Kk_1 (1+\frac{s}{a})}{s(1+\frac{s}{b})\left(1+\frac{s}{(\frac{1}{T_2})}\right)\left(1+\frac{s}{(\frac{1}{T_1})}\right)}\]
2) Lågfrekvensasymptot: 
\[s \to 0 \Rightarrow G_0(s) \to \frac{Kk_1}{s} = \frac{0.05}{s}\]
3) Högfrekvensasymptot: 
\[s \to \infty \Rightarrow G_0(s) \to \frac{Kk_1 \cdot \frac{s}{a} }{s \cdot \frac{s}{b} \cdot \frac{s}{1/T_2} \cdot \frac{s}{1/T_1}} = \frac{Kk_1b}{aT_1T_2} \cdot \frac{1}{s^3} = \frac{1.25 \cdot 10^{-4}}{s^3}\]
4)+5) Brytpunkter + bidrag 
\begin{table}[htbp]
  \centering
  \begin{tabular}{c|c|c|c|c|c}
 Punkt & $0$ & $1/T_1=0.01$ & $a=0.02$ & $b=0.05$ & $1/T_2=0.1$ \\
 Typ & pol & pol & nollställe & pol & pol \\
 Bidrag & -1 dek/dek & -1 dek/dek & +1 dek/dek & -1 dek/dek & -1 dek/dek \\
 Lutning & -1 dek/dek & -2 dek/dek & -1 dek/dek & -2 dek/dek & -3 dek/dek 
  \end{tabular}
  \caption{Brytpunkterna till $G_0$ med lutningsbidrag och resulterande lutning}
  \label{tab:brytbidr}
\end{table}
\FloatBarrier
Resultatet stämmer överens med låg- och högfrekvensasymptoterna. \\
6) Förankra \\
Välj en godtyckligt punkt mellan första och andra brytpunkt (eller mellan 0 och första brytpunkt ifall $p = 0$). 
\[|G_0|_{lf} = \lbrace \omega = 0.005 \rbrace = \frac{0.05}{0.005} = 10\]
7) Beräkna några punkter på faskurvan
\begin{table}[htbp]
  \centering
  \begin{tabular}{c|c|c|c|c|c|c}
 Frekvens & $0.001$ & $0.005$ & $0.01$ & $0.02$ & $0.04$ & $0.08$ \\
 Fas & $-95 ^{\circ}$ & $-111 ^{\circ}$ & $-125 ^{\circ}$ & $-142 ^{\circ}$ & $-163 ^{\circ}$ & $-194^{\circ}$
  \end{tabular}
  \caption{Fasen för $G_0(s)$ för några $\omega$}
  \label{tab:fas}
\end{table}
\FloatBarrier
Det resulterande bodediagrammet återges i Figur \ref{fig:42bode}. Notera att det verkliga diagrammet befinner sig något under asymptoterna.
\begin{figure}[h!]
  \centering
  \includegraphics[width=0.8\textwidth]{4_2bode}
  \caption{Bodediagram för $G_0(s)$}
  \label{fig:42bode}
\end{figure}
\FloatBarrier

\subsection*{b)}
$K$ ökar till dess systemet börjar självsvänga. Vid vilket $K$ värde sker detta och med vilken period sker svängningarna?
\qline
Systemet börjar självsvänga precis när det övergår till instabilitet, det vill säga då Nyquistkurvan korsar $-1$. I Bodediagrammet motsvarar detta att $\omega_c = \omega_p$ eftersom Nyquistkurvan då skär negativa Re-axeln med belopp $1$. Från Bodediagrammet erhålls 
\[|0.5 G(i\omega)| = \lbrace \omega = \omega_p = 0.06 \rbrace = 0.24\]
så att 
\[|G(i\omega_p)| = 0.48\]
Välj $K$ så att $\omega_c = \omega_p$: 
\[K |G(i\omega_p= | = 1 \Rightarrow K = \frac{1}{0.48} = 2.1\]
Vid detta $K$ är $\omega_c = \omega_p$ så att systemet börjar självsvänga med frekvens $\omega_c = \omega_p = 0.06$, vilket motsvarar perioden 
\[T = \frac{2\pi}{\omega_c} = \frac{2\pi}{0.06} = 105 \mathrm{s}\]

\section*{Bodediagram för slutna system}
I Figur \ref{fig:closedbode} återgivs Bodediagrammet för ett slutet system
\begin{figure}[h!]
  \centering
  \includegraphics[width=0.8\textwidth]{closedbode}
  \caption{Bodediagram för ett slutet system (Källa: Mariette Annergren)}
  \label{fig:closedbode}
\end{figure}
\FloatBarrier

\begin{itemize}
\item $M_p$ - Resonanstopp \\
Den maximala förstärkningen som uppnås med $G_c(s)$
\item $\omega_r$ - Resonansfrekvens \\
Signaler med frekvens $\omega_r$ förstärks mest. I princip gäller 
\[M \sim M_p\]
där $M$ är stegsvarets översläng
\item $\omega_B$ - Bandbredd \\
  Frekvens då $|G_c(s)| = \dfrac{1}{\sqrt{2}}$, säger något om vilka frekvenser som systemet förstärker och därmed hur snabbt systemet är. I princip gäller 
  \[\omega_B \sim \frac{1}{T_r}\]
  där $T_r$ är stegsvarets stigtid
\item $|G_c(0)|$ - Statisk förstärkning \\
Ifall $|G_c(0)| = 1$ kommer stegsvaret inte ha något stationärt fel
\end{itemize}

\section*{Uppgift 4.4}
Para ihop rätt stegsvar med rätt Bodediagram i Figur \ref{fig:4_4bode}
\begin{figure}[h!]
  \centering
  \includegraphics[width=0.6\textwidth]{4_4bode}
  \caption{Fyra stegsvar med tillhörande Bodediagram i okänd ordning}
  \label{fig:4_4bode}
\end{figure}
\FloatBarrier

\begin{itemize}
\item B har stationärt fel och endast 3) har $|G_c(0)| \neq 1$ \\
  $\Rightarrow B \leftrightarrow 3$
\item D har lägst översläng och 1) har lägst resonanstopp \\
  $\Rightarrow D \leftrightarrow 1$
\item C är långsammare än A (lägre $T_r$) och 4) har lägre bandbredd än 2) \\
  $\Rightarrow C \leftrightarrow 4, \;\;\; A \leftrightarrow 2$
\end{itemize}
Sammanfattningsvis:
\begin{align*}
  B &\leftrightarrow 3 \\
  D &\leftrightarrow 1 \\
  C &\leftrightarrow 4 \\
  A &\leftrightarrow 2
\end{align*}

\section*{Uppgift 5.8}
Ett blockdiagram för ett system med en tidsfördröjning är återgivet i Figur \ref{fig:5_8block}
\begin{figure}[h!]
  \centering
  \includegraphics[width=0.6\textwidth]{5_8block}
  \caption{System med tidsfördröjning}
  \label{fig:5_8block}
\end{figure}
\FloatBarrier
$G_1(s)$ har inga poler i HHP.

\subsection*{a)}
Om $G_1(s)$ har Bodediagrammet som återges i Figur \ref{fig:5_8bode}
\begin{figure}[h!]
  \centering
  \includegraphics[width=0.6\textwidth]{5_8bode}
  \caption{Bodediagram för $G_1(s)$}
  \label{fig:5_8bode}
\end{figure}
\FloatBarrier
För vilka $T$ blir det slutna systemet stabilt?
\qline
Hur påverkar tidsfördröjningen Bodediagrammet? \\

\textbf{Amplitud:} 
\[|e^{-i\omega T}| = 1\]
$\Rightarrow$ amplituden påverkas inte! \\

\textbf{Fas:} 
\[\arg{e^{-i\omega T}} = -\omega T\]
$\Rightarrow$ Fasen minskar med $\omega T$! \\

Ur Bodediagrammet urläses
\begin{align*}
  \omega_c &= 1 \mathrm{rad/s} \\
  \varphi_m &= 40 ^{\circ} = 0.698 \mathrm{rad}
\end{align*}
Systemet är stabilt ifall fasmarginalen fortfarande är positiv efter tidsfördröjningens påverkan. 
\begin{align*}
  \varphi_m^* &= \varphi_m - \omega_c T \\
  &= 0.698-1 \cdot T
\end{align*}
Alltså krävs
\begin{align*}
  0.698-T > 0 \\
  \Rightarrow T &< 0.698
\end{align*}










\end{document}

