\documentclass[12pt]{article}
\usepackage[swedish]{babel}
\usepackage[T1]{fontenc}{
\usepackage[utf8]{inputenc}
\usepackage{amssymb}
\usepackage{amsmath}
\usepackage{amsfonts}
\usepackage[colorinlistoftodos,prependcaption,textsize=small]{todonotes}
\usepackage{url,graphicx,tabularx,array,geometry,color,float,placeins}
\setlength{\parskip}{1ex} %--skip lines between paragraphs
\setlength{\parindent}{0pt} %--don't indent paragraphs
%\setcounter{secnumdepth}0}
\usepackage{tikz}
\usetikzlibrary{shapes, arrows, 3d, decorations, calc, decorations.markings}

\tikzstyle{block} = [draw, rectangle, thick, minimum height=3em, minimum width=2em, align=center, text width=1.5cm, top color=white, bottom color=white!85!black]
\tikzstyle{input} = [coordinate]
\tikzstyle{output} = [coordinate]
\tikzstyle{tmp} = [coordinate]
\tikzstyle{sum} = [draw, circle, node distance=1cm]
\tikzstyle{wire} = [->,thick]
\tikzstyle{lab} = []
\tikzstyle{laba} = [lab, label distance=0cm]
\tikzstyle{labb} = [lab, label distance=0.5em]
\tikzstyle{split} = [fill,black,circle,inner sep=0.03cm]
\tikzset{cross/.style={cross out, draw=black, minimum size=2*(#1-\pgflinewidth), inner sep=0pt, outer sep=0pt},
%default radius will be 1pt. 
cross/.default={5pt}}

\newcommand{\sspan}[1]{\mathrm{span}\left\{#1\right\}}
\newcommand{\ima}[1]{\mathrm{Im}\; #1}
\newcommand{\qline}{\hrule \vspace*{10pt}}

%\linespread{2} %-- Uncomment for Double Space
\begin{document}
\begin{titlepage}
\author{Martin Biel \\ \texttt{mbiel@kth.se}}
\title{EL1000 - Reglerteknik, allmän kurs \\ \Large Övning 10}
\date{27 september 2016}
\end{titlepage}

\maketitle

\section*{Repetition}
\begin{align*}
  \dot{x} &= Ax + Bu \\
  y &= Cx + Du
\end{align*}
\textbf{Styrbarhet} \\
\begin{itemize}
\item Ett tillstånd $x^*$ är \emph{styrbart} om det finns en insignal $u(t)$ som tar tillståndsvektorn från $x(0) = 0$ till $x^*$ på ändlig tid.
\item $\mathcal{S} =
  \begin{bmatrix}
    B & AB & \dots & A^{n-1}B
  \end{bmatrix}$ - Styrbarhetsmatrisen
\item De styrbara tillstånden ligger i $\sspan{\mathcal{S}}$.
\item Systemet är \emph{styrbart} om alla tillstånd är styrbara, dvs om $\sspan{\mathcal{S}} = \mathbb{R}^n$ (alternativt ifall $\mathcal{S}$ har full rang, eller ifall $\det{\mathcal{S}} \neq 0$ om $\mathcal{S}$ är kvadratisk.
\end{itemize}
\textbf{Observerbarhet}
\begin{itemize}
\item Ett tillstånd $x^* \neq 0$ är \emph{icke-observerbart} om utsignalen är identiskt noll ($y(t) \equiv 0$) då $x(0) = x^*$ och $u(t) \equiv 0$.
\item $\mathcal{O} =
  \begin{bmatrix}
    C \\
    CA \\
    \vdots \\
    CA^{n-1}
  \end{bmatrix}$ - Observerbarhetsmatrisen
\item De icke-observerbara tillstånden ligger i $\ker{\mathcal{O}}$ (nollrummet till $\mathcal{O}$).
\item Systemet är \emph{observerbart} om det saknar icke-observerbara tillstånd, dvs $\ker{\mathcal{O}} = \emptyset$ (alternativt ifall $\sspan{\mathcal{O}} = \mathbb{R}^n$, $\mathcal{O}$ har full rang, eller ifall $\det{\mathcal{O}} \neq 0$ om $\mathcal{O}$ är kvadratisk).
\end{itemize}

\section*{Uppgift 8.10}
Bestäm dimenion för de styrbara och icke-observerbara underrummen. Bestäm även dessa underrum.

\subsection*{a)}
\begin{align*}
  \dot{x} &=
            \begin{pmatrix}
              -2 & 0 & 0 \\
              0 & -1 & 1 \\
              0 & 0 & -3
            \end{pmatrix}x +
                      \begin{pmatrix}
                        1 \\
                        -1 \\
                        2
                      \end{pmatrix}u \\
  y &=
      \begin{pmatrix}
        1 & 3 & 1.5
      \end{pmatrix}x
\end{align*}
\qline
\textbf{Styrbarhet:}
\[\mathcal{S} =
\begin{pmatrix}
  1 & -2 & 4 \\
  -1 & 3 & -9 \\
  2 & -6 & 18
\end{pmatrix}\]
Första och andra raden är linjärt beroende. Ifall en stryks är de återstående linjärt oberoende.\\
$\Rightarrow \dim{(\sspan{\mathcal{S}})} = 2$.
Det styrbara underrummet spänns upp av två valfria linjärt oberoende kolumner i $\mathcal{S}$:
\begin{equation*}
  \sspan{\mathcal{S}} = \sspan{
    \begin{pmatrix}
      1 \\
      -2 \\
      2
    \end{pmatrix},
    \begin{pmatrix}
      -2 \\
      3 \\
      -6
    \end{pmatrix}}
\end{equation*}
\textbf{Observerbarhet:}
\begin{equation*}
  \mathcal{O} =
  \begin{pmatrix}
    1 & 3 & 1.5 \\
    -2 & -3 & -1.5 \\
    4 & 3 & 1.5
  \end{pmatrix}
\end{equation*}
Ifall vi utför Gauss-elíminering kommer vi kunna svara på båda frågeställningarna.
\begin{equation*}
  \left(
  \begin{array}{ccc|c}
    1 & 3 & 1.5 & 0\\
    -2 & -3 & -1.5 & 0 \\
    4 & 3 & 1.5 & 0
  \end{array}\right) ~ 
\left(
  \begin{array}{ccc|c}
    1 & 3 & 1.5 & 0 \\
    0 & 3 & 1.5 & 0 \\
    0 & -9 & -4.5 & 0
  \end{array}
\right) ~ \left(
  \begin{array}{ccc|c}
    1 & 3 & 1.5 & 0 \\
    0 & 3 & 1.5 & 0 \\
    0 & 0 & 0 & 0
  \end{array}
\right)
\end{equation*}
$\Rightarrow \dim{(\sspan{\mathcal{O}})} = 2, \dim{(\ker{\mathcal{O}})} = 1$ \\
En basvektor för $\ker{\mathcal{O}}$ bestäms genom att införa $x_3 = t, t \in \mathbb{R}$ vilket ger $x_2 = -0.5t$ och $x_1 = 0$. Alltså ges nollrummet av
\[\Rightarrow \ker{\mathcal{O}} = \sspan{
  \begin{pmatrix}
    0 \\
    -0.5 \\
    1
  \end{pmatrix}
}\]
vilket även ger underrummet av icke-observerbara tillstånd.

\subsection*{b)}

\begin{align*}
  \dot{x} &=
            \begin{pmatrix}
              -1 & 0 & 0 \\
              1 & -2 & 0 \\
              0 & 0 & -4
            \end{pmatrix}x +
                      \begin{pmatrix}
                        0 \\
                        4 \\
                        -2
                      \end{pmatrix}u \\
  y &=
      \begin{pmatrix}
        0 & 3 & 0
      \end{pmatrix}x
\end{align*}
\qline
\textbf{Styrbarhet:}
\[\mathcal{S} =
\begin{pmatrix}
  0 & 0 & 0 \\
  4 & -8 & 16 \\
  -2 & 8 & -32
\end{pmatrix}\]
Första raden faller bort och återstående rader är linjärt oberoende.\\
$\Rightarrow \dim{(\sspan{\mathcal{S}})} = 2$.
Det styrbara underrummet spänns upp av två valfria linjärt oberoende kolumner i $\mathcal{S}$:
\begin{equation*}
  \sspan{\mathcal{S}} = \sspan{
    \begin{pmatrix}
      0 \\
      4 \\
      -2
    \end{pmatrix},
    \begin{pmatrix}
      0 \\
      -8 \\
      8
    \end{pmatrix}}
\end{equation*}
\textbf{Observerbarhet:}
\begin{equation*}
  \mathcal{O} =
  \begin{pmatrix}
    0 & 3 & 0 \\
   3 & -6 & 0 \\
    -9 & 12 & 0
  \end{pmatrix}
\end{equation*}
Tredje kolumnen faller bort och de återstående är linjärt oberoende. \\
$\Rightarrow \dim{(\sspan{\mathcal{O}})} = 2, \dim{(\ker{\mathcal{O}})} = 1$. Utför Gauss-eliminering:
\begin{equation*}
  \left(
  \begin{array}{ccc|c}
    0 & 3 & 0 & 0\\
    3 & -6 & 0 & 0 \\
    -9 & 12 & 0 & 0
  \end{array}\right) ~ 
\left(
  \begin{array}{ccc|c}
    3 & -6 & 0 & 0 \\
    0 & 3 & 0 & 0 \\
    0 & -12 & 0 & 0
  \end{array}
\right)
\end{equation*}
$\Rightarrow x_1 = 0, x_2 = 0$ och $x_3 = t, t \in \mathbb{R}$. \\
$\Rightarrow \dim{(\ker{\mathcal{O}})} = 1$ \\
Alltså ges nollrummet av
\[\Rightarrow \ker{\mathcal{O}} = \sspan{
  \begin{pmatrix}
    0 \\
    0 \\
    1
  \end{pmatrix}
}\]
vilket även ger underrummet av icke-observerbara tillstånd.

\section*{Uppgift 8.13}

\begin{figure}[h!]
  \centering
  \includegraphics[width=0.5\textwidth]{8_13pend}
  \caption{Two pendlar på en vagn}
  \label{fig:813pend}
\end{figure}

\subsection*{a)}
Linjärisera systemet kring $\varphi_1 = \varphi_2 = 0$ när $l = m = g = 1$.
\qline
Pendel 1: $\ddot{z}\cos{\varphi_1} + \ddot{\varphi}_1 = \sin{\varphi_1}$ \\
Pendel 2: $\ddot{z}\cos{\varphi_2} + \alpha\ddot{\varphi}_2 = \sin{\varphi_2}$ \\
Linjärisering: \\
Pendel 1: $\ddot{z} + \ddot{\varphi}_1 = \varphi_1$ \\
Pendel 2: $\ddot{z} + \alpha\ddot{\varphi}_2 = \varphi_2$ \\
Låt $u = \ddot{z}$ utgöra insignal och inför tillstånden
\begin{align*}
x_1 &= \varphi_1 &\Rightarrow \dot{x}_1 &= x_2 \\
x_2 &= \dot{\varphi}_1 &\Rightarrow \dot{x}_2 &= x_1 - u\\
x_3 &= \varphi_2 &\Rightarrow \dot{x}_3 &= x_4\\
x_4 &= \dot{\varphi}_2 &\Rightarrow \dot{x}_4 &= \frac{1}{\alpha}(x_3 - u)
\end{align*}
Alltså erhålls
\[\dot{x} = \begin{pmatrix}
0 & 1 & 0 & 0 \\
1 & 0 & 0 & 0 \\
0 & 0 & 0 & 1 \\
0 & 0 & \frac{1}{\alpha} & 0
\end{pmatrix}x + \begin{pmatrix}
0 \\
-1 \\
0 \\
-\frac{1}{\alpha}
\end{pmatrix}u \]
\subsection*{b)}
För vilka $\alpha$ är systemet styrbart?
\qline
Styrbarhetsmatrisen ges av
\[\mathcal{S} = \begin{pmatrix}
0 & -1 & 0 & -1 \\
-1 & 0 & -1 & 0 \\
0 & -\frac{1}{\alpha} & 0 & -\frac{1}{\alpha^2} \\
-\frac{1}{\alpha} & 0 & -\frac{1}{\alpha^2} & 0
\end{pmatrix} \]
$\mathcal{S}$ är kvadratisk, så vi kan beräkna determinanten för att dra slutsatser om styrbarhet.
\begin{align*}
{\mathcal{S}} &= \begin{vmatrix}
-1 & -1 & 0 \\
0 & 0 & -\frac{1}{\alpha^2} \\
-\frac{1}{\alpha} & -\frac{1}{\alpha} & 0
\end{vmatrix} + \begin{vmatrix}
-1 & 0 & -1 \\
0 & -\frac{1}{\alpha} & 0 \\
-\frac{1}{\alpha} & 0 & -\frac{1}{\alpha^2}
\end{vmatrix} \\
&= \frac{1}{\alpha^2}\begin{vmatrix}
-1 & -1 \\
-\frac{1}{\alpha} & -\frac{1}{\alpha^2}
\end{vmatrix} - \begin{vmatrix}
-\frac{1}{\alpha} & 0 \\
0 & -\frac{1}{\alpha^2}
\end{vmatrix} - \begin{vmatrix}
0 & -\frac{1}{\alpha} \\
-\frac{1}{\alpha} & 0
\end{vmatrix} \\
&= \frac{1}{\alpha^2}\left(\frac{1}{\alpha^2} - \frac{1}{\alpha}\right) - \frac{1}{\alpha^3} + \frac{1}{\alpha^2} \\
&= \frac{1}{\alpha^2}\left(\frac{1}{\alpha^2} - \frac{2}{\alpha} + 1\right) \\
&= \frac{1}{\alpha^2}\left(\frac{1}{\alpha} - 1\right)^2 \\
&= \left(\frac{1-\alpha}{\alpha^2}\right)^2
\end{align*}
$\Rightarrow \det{\mathcal{S}} \neq 0$ om $\alpha \neq 1$. \\
$\Rightarrow$ Systemet är styrbart ifall $\alpha \neq 1$. \\\\
Fallet $\alpha = 1$ motsvarar ett system där pendlarna är lika långa, vilket resulterar i att  varje insignal ger upphov till samma vinkelförändring $\varphi_1 = \varphi_2$ i båda pendlarna. Således går det inte att uppnå alla tillstånd eftersom det inte är möjligt att nå tillstånd där $\varphi_1 \neq \varphi_2$.\\

\section*{Tillståndsåterkoppling}
\begin{figure}[h!]
  \centering
\begin{tikzpicture}[auto, node distance=2cm, >=latex']
\node [block, text width=3cm, text height = 0.1cm] (system) {
  \begin{align*}
    \dot{x} &= Ax + Bu \\
    y &= Cx
  \end{align*}
};
\node [sum, left of=system, node distance=3cm] (sum) {};
\node [block, below of=system, node distance=3cm] (L){$L$};

\draw [wire] ++(-6,0) -- node[pos=0.99] {$+$} (sum) node[pos=0.45, right, align=center, label={[laba]below:$y_{ref}(t)$}] {};
\draw [wire] (sum) -- (system) node[pos=0.45, right, align=center, label={[laba]below:$u(t)$}] {};
\draw [wire] (system) -- ++(4,0) node[pos=0.45, right, align=center, label={[laba]below:$y(t)$}] {};
\draw [wire] (system) -- node[pos=0.5, right] {$x(t)$} (L);
\draw [wire] (L) -| node[pos=0.99, right] {$-$} (sum);
\end{tikzpicture}
  \caption{Tillståndsåterkoppling}
  \label{fig:återk}
\end{figure}
Återkoppla systemets tillstånd istället för utsignalen. Slutna systemet ges av
\begin{align*}
\dot{x} &= Ax + B(r-Lx) = (A-BL)x + Br \
y &= Cx
\end{align*}
$\Rightarrow$ Slutna systemet får poler motsvarande egenvärdena till $A-BL$ (istället för $A$). \\
$\Rightarrow$ Vi kan påverka polerna med $L$!\\

\textbf{Sats} (s.183) \\
Ifall systemet är styrbart kan polerna (egenvärdena) till $A-BL$ placeras godtyckligt.\\

Ifall systemet inte är styrbart kan det ändå vara möjligt att placera polerna önskvärt.

\section*{Tillståndsobservatör}
Om inte tillstånden är mätbara kan en observatör användas för att uppskatta $x(t)$ från $y(t)$.
\begin{figure}[h!]
  \centering
\begin{tikzpicture}[auto, node distance=2cm, >=latex']
\node [block, text width=3cm, text height = 0.1cm] (system) {
  \begin{align*}
    \dot{x} &= Ax + Bu \\
    y &= Cx
  \end{align*}
};
\node [block, below of=system, node distance=3cm, text width=2cm] (obs) {Observatör};
\node [sum, left of=system, node distance=5cm] (sum) {};
\node [split, right of=system] (split2) {};
\node [split, left of=system] (split1) {};
\node [block, left of= obs, node distance=3cm] (L){$L$};
\node [tmp, below of=system, node distance=1.5cm] (tmp){};

\draw [wire] ++(-7,0) -- node[pos=0.99] {$+$} (sum) node[pos=0.4, right, align=center, label={[laba]below:$y_{ref}(t)$}] {};
\draw [wire] (sum) -- (system) node[pos=0.45, right, align=center, label={[laba]below:$u(t)$}] {};
\draw [wire] (system) -- ++(4,0) node[pos=0.45, right, align=center, label={[laba]below:$y(t)$}] {};
\draw [wire] (split1) |- (tmp) -- (obs);
\draw [wire] (split2) |- (obs);
\draw [wire] (obs) -- (L) node[pos=0.5, above] {$\hat{x}$};
\draw [wire] (L) -| node[pos=0.99, right] {$-$} (sum);
\end{tikzpicture}
  \caption{Tillståndsobservatör}
  \label{fig:återk}
\end{figure}
Observatören är också ett dynamiskt system som "simulerar" det ursprungliga systemet.
\[\dot{\hat{x}} = A\hat{x} + Bu + K(y-C\hat{x})\]
De två första termerna simulerar systemet och den sista termen är en korrigerande term som motsvarar återkoppling av den verkliga utsignalen (observatören simulerar $C\hat{x}$ men utsignalen ges av $y$). Undersök skattningsfelet $\tilde{x} = x - \hat{x}$:s dynamik:
\begin{align*}
\dot{\tilde{x}} &= Ax + Bu - \lbrace A\hat{x} + Bu + K(y-C\hat{x}) \rbrace \\
&= Ax - A\hat{x} + K(Cx - C\hat{x}) \\
&= (A-KC)\tilde{x}
\end{align*}
$\Rightarrow$ Om $(A-KC)$ är stabil (egenvärden i VHP) så går skattningsfelet mot noll. \\

\textbf{Sats} (s.193) \\
Ifall systemet är observerbart kan polerna (egenvärdena) till $A-KC$ placeras godtyckligt\\

Ifall systemet inte är observerbart kan det ändå vara möjligt att placera polerna önskvärt.

\section{Uppgift 9.1}

\begin{align*}
\dot{x} &= \begin{pmatrix}
-2 & -1 \\
1 & 0
\end{pmatrix}x + \begin{pmatrix}
1 \\
0
\end{pmatrix}u \\
y &= \begin{pmatrix}
1 & 0
\end{pmatrix}x
\end{align*}

\subsection*{a)}
Bestäm en tillståndsåterkoppling som placerar polerna i
\begin{itemize}
\item[(i)] $-3,-5$ \\
\item[(ii)] $-10, -15$
\end{itemize} 
Vad begränsar godtycklig dynamik?
\qline
\[\mathcal{S} = \begin{pmatrix}
1 & -2 \\
0 & 1
\end{pmatrix} \]
har full rang. \\
$\Rightarrow$ Kan placera polerna till $A-BL$ godtyckligt.
\[A-BL = \begin{pmatrix}
-2 & -1 \\
1 & 0
\end{pmatrix} - \begin{pmatrix}
1 \\
0
\end{pmatrix} \begin{pmatrix}
l_1 & l_2
\end{pmatrix} = \begin{pmatrix}
-2-l_1 & -1-l_2 \\
1 & 0
\end{pmatrix} \]

\[\det{(sI - (A-BL)} = 0 \Leftrightarrow \begin{vmatrix}
s + 2 + l_1 & 1 + l_2 \\
-1 & s
\end{vmatrix} = 0 \]
\begin{align*}
&\Leftrightarrow s(s+2+l_1= + 1 + l_2 = 0 \\
&\Leftrightarrow s^2 + (2 + l_1)s + (1 + l_2) = 0
\end{align*}
Identifiera koefficient så att de efterfrågade polerna erhålls. \\

\textbf{(i)} $(s+3)(s+5) = s^2 + 8s + 15$ \\
$\Rightarrow l_1 = 6, l_2 = 14$ \\
$\Rightarrow u = -6x_1 - 14x_2 + y_{ref}$ \\

\textbf{(ii)} $(s+10)(s+15) = s^2 + 25s + 150$ \\
$\Rightarrow l_1 = 23, l_2 = 149$ \\
$\Rightarrow u = -23x_1 - 149x_2 + y_{ref}$ \\

Ju längre ut i VHP polerna placeras desto  större koefficienter dyker upp i styrlagen, vilket resulterar i en större insignal. \\
$\Rightarrow$ Godtycklig dynamik kan ej uppnås på grund av fysiska begränsningar i insignalen. 

\subsection*{b)}
Om enbart $y(t)$ kan mätas, bestäm en observatör. (och diskutera polernas påverkan)
\qline
\[\mathcal{O} = \begin{pmatrix}
1 & 0 \\
-2 & -1
\end{pmatrix} \]
har full rang. \\
$\Rightarrow$ kan placera polerna till $A-KC$ godtyckligt. \\
Skattningsfelets dynamik ges av
\[\dot{\tilde{x}} = (A-KC)\tilde{x}\]
Vi vill att skattningsfelet går mot noll snabbare än det styrda systemets dynamik. Således bör observatörens poler placeras längre till vänster i VHP än det slutna systemet. (t.ex. i $s = -20$). Ifall polerna placeras för långt ut blir observatören känslig för mätbrus.
\[A - KC = \begin{pmatrix}
-2 & -1 \\
1 & 0
\end{pmatrix} - \begin{pmatrix}
k_1 \\
k_2
\end{pmatrix} \begin{pmatrix}
1 & 0
\end{pmatrix} = \begin{pmatrix}
-2 -k_1 & -1 \\
1 - k_2 & 0
\end{pmatrix} \]
\[\det{(sI-(A-KC))} = 0\]
\begin{align*}
&\Leftrightarrow \begin{vmatrix}
s + (2+k_1) & 1 \\
-1+k_2 & s
\end{vmatrix} = 0 \\
&\Leftrightarrow s^2 + (2+k_1)s + (1-k_2) = 0
\end{align*}
Två poler i $s = -20$ ger
\[(s+20)^2 = s^2 + 40s + 400\]
$\Rightarrow k_1 = 38, k_2 = -399$
Den resulterande observatören ges av
\[\dot{\hat{x}} = \begin{pmatrix}
-2 & -1 \\
1 & 0
\end{pmatrix}\hat{x} + \begin{pmatrix}
1 \\
0
\end{pmatrix}u + \begin{pmatrix}
38 \\
-399
\end{pmatrix} [y - \begin{pmatrix}
1 & 0
\end{pmatrix}\hat{x}] \]



\end{document}

