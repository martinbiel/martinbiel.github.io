\documentclass[12pt]{article}
\usepackage[swedish]{babel}
\usepackage[T1]{fontenc}{
\usepackage[utf8]{inputenc}
\usepackage{amssymb}
\usepackage{amsmath}
\usepackage{amsfonts}
\usepackage[colorinlistoftodos,prependcaption,textsize=small]{todonotes}
\usepackage{url,graphicx,tabularx,array,geometry,color,float,placeins}
\setlength{\parskip}{1ex} %--skip lines between paragraphs
\setlength{\parindent}{0pt} %--don't indent paragraphs
%\setcounter{secnumdepth}0}
\usepackage{tikz}
\usetikzlibrary{shapes, arrows, 3d, decorations, calc, decorations.markings}

\tikzstyle{block} = [draw, rectangle, thick, minimum height=3em, minimum width=2em, align=center, text width=1.5cm, top color=white, bottom color=white!85!black]
\tikzstyle{input} = [coordinate]
\tikzstyle{output} = [coordinate]
\tikzstyle{tmp} = [coordinate]
\tikzstyle{sum} = [draw, circle, node distance=1cm]
\tikzstyle{wire} = [->,thick]
\tikzstyle{lab} = []
\tikzstyle{laba} = [lab, label distance=0cm]
\tikzstyle{labb} = [lab, label distance=0.5em]
\tikzstyle{split} = [fill,black,circle,inner sep=0.03cm]
\tikzset{cross/.style={cross out, draw=black, minimum size=2*(#1-\pgflinewidth), inner sep=0pt, outer sep=0pt},
%default radius will be 1pt. 
cross/.default={5pt}}

\newcommand{\sspan}[1]{\mathrm{span}\left\{#1\right\}}
\newcommand{\ima}[1]{\mathrm{Im}\; #1}
\newcommand{\qline}{\hrule \vspace*{10pt}}

%\linespread{2} %-- Uncomment for Double Space
\begin{document}
\begin{titlepage}
\author{Martin Biel \\ \texttt{mbiel@kth.se}}
\title{EL1000 - Reglerteknik, allmän kurs \\ \Large Övning 9}
\date{22 september 2016}
\end{titlepage}

\maketitle

\section*{Repetition}
\begin{align*}
  \dot{x} &= Ax + Bu \\
  y &= Cx + Du
\end{align*}
\textbf{Styrbarhet} \\
\begin{itemize}
\item Ett tillstånd $x^*$ är \emph{styrbart} om det finns en insignal $u(t)$ som tar tillståndsvektorn från $x(0) = 0$ till $x^*$ på ändlig tid.
\item $\mathcal{S} =
  \begin{bmatrix}
    B & AB & \dots & A^{n-1}B
  \end{bmatrix}$ - Styrbarhetsmatrisen
\item De styrbara tillstånden ligger i $\sspan{\mathcal{S}}$.
\item Systemet är \emph{styrbart} om alla tillstånd är styrbara, dvs om $\sspan{\mathcal{S}} = \mathbb{R}^n$ (alternativt ifall $\mathcal{S}$ har full rang, eller ifall $\det{\mathcal{S}} \neq 0$ om $\mathcal{S}$ är kvadratisk.
\end{itemize}
\textbf{Observerbarhet}
\begin{itemize}
\item Ett tillstånd $x^* \neq 0$ är \emph{icke-observerbart} om utsignalen är identiskt noll ($y(t) \equiv 0$) då $x(0) = x^*$ och $u(t) \equiv 0$.
\item $\mathcal{O} =
  \begin{bmatrix}
    C \\
    CA \\
    \vdots \\
    CA^{n-1}
  \end{bmatrix}$ - Observerbarhetsmatrisen
\item De icke-observerbara tillstånden ligger i $\ker{\mathcal{O}}$ (nollrummet till $\mathcal{O}$).
\item Systemet är \emph{observerbart} om det saknar icke-observerbara tillstånd, dvs $\ker{\mathcal{O}} = \emptyset$ (alternativt ifall $\sspan{\mathcal{O}} = \mathbb{R}^n$, $\mathcal{O}$ har full rang, eller ifall $\det{\mathcal{O}} \neq 0$ om $\mathcal{O}$ är kvadratisk).
\end{itemize}

\section*{Uppgift 8.10}
Bestäm dimenion för de styrbara och icke-observerbara underrummen. Bestäm även dessa underrum.

\subsection*{a)}
\begin{align*}
  \dot{x} &=
            \begin{pmatrix}
              -2 & 0 & 0 \\
              0 & -1 & 1 \\
              0 & 0 & -3
            \end{pmatrix}x +
                      \begin{pmatrix}
                        1 \\
                        -1 \\
                        2
                      \end{pmatrix}u \\
  y &=
      \begin{pmatrix}
        1 & 3 & 1.5
      \end{pmatrix}x
\end{align*}
\qline
\textbf{Styrbarhet:}
\[\mathcal{S} =
\begin{pmatrix}
  1 & -2 & 4 \\
  -1 & 3 & -9 \\
  2 & -6 & 18
\end{pmatrix}\]
Första och andra raden är linjärt beroende. Ifall en stryks är de återstående linjärt oberoende.\\
$\Rightarrow \dim{\sspan{\mathcal{S}}} = 2$.
Det styrbara underrummet spänns upp av två valfria linjärt oberoende kolumner i $\mathcal{S}$:
\begin{equation*}
  \sspan{\mathcal{S}} = \sspan{
    \begin{pmatrix}
      1 \\
      -2 \\
      2
    \end{pmatrix},
    \begin{pmatrix}
      -2 \\
      3 \\
      -6
    \end{pmatrix}}
\end{equation*}
\textbf{Observerbarhet:}
\begin{equation*}
  \mathcal{O} =
  \begin{pmatrix}
    1 & 3 & 1.5 \\
    -2 & -3 & -1.5 \\
    4 & 3 & 1.5
  \end{pmatrix}
\end{equation*}
Ifall vi utför Gauss-elíminering kommer vi kunna svara på båda frågeställningarna.
\begin{equation*}
  \left(
  \begin{array}{ccc|c}
    1 & 3 & 1.5 & 0\\
    -2 & -3 & -1.5 & 0 \\
    4 & 3 & 1.5 & 0
  \end{array}\right) ~ 
\left(
  \begin{array}{ccc|c}
    1 & 3 & 1.5 & 0 \\
    0 & 3 & 1.5 & 0 \\
    0 & -9 & -4.5 & 0
  \end{array}
\right) ~ \left(
  \begin{array}{ccc|c}
    1 & 3 & 1.5 & 0 \\
    0 & 3 & 1.5 & 0 \\
    0 & 0 & 0 & 0
  \end{array}
\right)
\end{equation*}
$\Rightarrow \dim{\sspan{\mathcal{O}}} = 2, \dim{\ker{\mathcal{O}}} = 1$ \\
En basvektor för $\ker{\mathcal{O}}$ bestäms genom att införa $x_3 = t, t \in \mathbb{R}$ och substituera.
\[\Rightarrow \ker{\mathcal{O}} = \sspan{
  \begin{pmatrix}
    0 \\
    -0.5 \\
    1
  \end{pmatrix}
}\]
vilket även ger underrummet av icke-observerbara tillstånd.

\subsection*{b)}


\end{document}

